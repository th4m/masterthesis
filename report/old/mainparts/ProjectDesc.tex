\chapter{PRJCTDESC}
\section{Project Description}

This project will create a compiler that takes code from a lazy programming
language and compiles it into a strict programming language. The resulting
strict code should have the same user-observable evaluation as the given lazy
code. The purpose of creating such a compiler is to explore the formal
connection between lazy and strict evaluation. The compiler will be written in
the programming language Haskell. The source language for the compiler will be
CakeML with custom semantics for laziness and the target language to which
the compiler will output will be CakeML with its standard strict semantics. This
means that the semantics to express laziness will be defined as part of this
project and translated to the original semantics of CakeML.

Internally, the compiler will take a
CakeML function with lazy semantics and translate it to one
or more strict CakeML functions in order to get a semantically equivalent
evaluation. When it comes to handling laziness, lazy functions will be
translated to explicit thunk values\cite{Ingerman:1961:TWC:366062.366084} in
CakeML. Thunks are expressions that have yet to be evaluated. Thunk values are
not primitive in CakeML. Our compiler will create code that
implements the thunk values in CakeML. The compiler will target the stateful
features of CakeML to avoid repeated evaluation of the same thunk values.

The operational semantics of lazy CakeML will be written as a part of the
project. As the semantics of CakeML is specified in higher-order logic (HOL)
\cite{HOLInter57:online},
the operational semantics for CakeML will be translated from HOL to Haskell
functions. The compiler will then connect the lazy and strict semantics.

Tests will be written to show that the compiler generates semantically
equivalent code. This project will employ manual unit testing to test
expressions and steps of evaluation.
A interesting and different approach to testing the compiler would be to
use property-based checking, with the automated testing library QuickCheck for
Haskell \cite{Introduc44:online}. In contrast to
unit-testing, which takes user-given data for its tests, QuickCheck does
property-based testing; it generates random test data according to the provided
specification and uses them to test properties. If time allows for it, this 
project will investigate
the possibility of using QuickCheck to test a compiler's formal correctness.

In summary, the project will consist of three major parts that will be developed
in parallel:
\begin{itemize}
 \item Writing lazy semantics for the programming language CakeML
 \item Creating a compiler that translates the lazy semantics to the original strict semantics of CakeML
 \item Testing the compiler by writing operational semantics in the form of functions and running unit tests, and possibly QuickCheck.
\end{itemize}
