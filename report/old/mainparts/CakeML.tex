\chapter{CakeML}

CakeML is a functional language based on a subset of
Standard ML~\cite{CakeML25:online}. It has a
formally verified compiler specified in higher-order logic that passes through
12 intermediate languages and targets machine code for 5 architectures.
%The language includes datatypes, pattern-matching, higher-order functions,
%references, exceptions, polymorphism, and modules and
%signatures~\cite{Kumar:2014:CVI:2535838.2535841}.

\begin{itemize}
\item How formally correct/verified?
\item Short introduction and discussion about the AST?
%% \item Strict semantics
%%   \begin{itemize}
%%   \item Environment and State
%%   \end{itemize}
\end{itemize}

The subset of CakeML that has been used for this project mainly
consists of expressions and values. Declarations were excluded as a result of
lack of time. Certain operations were also cut. One of these was bitstring
operators, mostly because of difference in implementation between CakeML
and Haskell, as well as it not being necessary for the proof of concept of
the compiler.
The basic grammar of expressions is shown in figure~\ref{CakeExp}. The
complementary grammar for expressions can be found in appendix~\ref{app:ASTExp}.

\begin{figure}
\begin{alltt}
Exp
  = Raise Exp
  | Handle Exp [(Pat, Exp)]
  | Literal Lit
  | Con (Maybe (Id ConN)) [Exp]
  | Var (Id VarN)
  | Fun VarN Exp
  | App Op [Exp]
  | Log LOp Exp Exp
  | If Exp Exp Exp
  | Mat Exp [(Pat, Exp)]
  | Let (Maybe VarN) Exp Exp
  | LetRec [(VarN, VarN, Exp)] Exp
  | TAnnot Exp T
\end{alltt}
\caption{Expressions of CakeML}
\label{CakeExp}
\end{figure}

\noindent Using this grammar, we can create expressions such as \texttt{take} from
section~\ref{intro:Example}:

[INSERT EXAMPLE]

%% \begin{figure}[H]
%% \begin{alltt}
%% LetRec [((Just "take"),(Just "n")
%%         ,If (App Equality [(Var (Short "n")),Literal (IntLit 0)])
%%           Con (Just "nil") []
%%           App OpApp )
%%        ,((Just "pats"),(Just "e")
%%          ,Mat (Var (Short "e"))
%%            [(PCon (Just "::")  [(PVar "l"), (PVar "ls")]
%%              ,Con (Just "::") [Var (Short "l")
%%                                , App OpApp [Var (Short "take")
%%                                            , Var (Short "ls")]])
%%            ,(PCon (Just "nil") []
%%              ,Con (Just "nil") [])])])
%% ]
%% \end{alltt}
%% \end{figure}

\section{Strict Semantics}
The first part of the project was to implement the existing strict semantics
of CakeML in Haskell. Besides writing the AST as described in
figure~\ref{CakeExp}, the \texttt{evaluate} function was implemented. This also
required the implementation of the underlying primitive operations, such as
arithmetics. \texttt{evaluate} has the following arguments: a list of
expressions, already described above, a \texttt{State}, and an
\texttt{Environment}. A \texttt{State} is a record type of data structure that
contains references, types, and module names. An \texttt{Environment} is also a
record type of data structure, that contains variables, constructor definitions,
and modules.

\texttt{evaluate} returns a result of type \texttt{Result [V] V}. This is a
flexible type that is either a list of \texttt{V} or an error. The data type
of \texttt{Result} is defined as follows:
\begin{figure}[H]
\begin{alltt}
data Result a b
  = RVal a
  | RErr (Error_Result b)
\end{alltt}
\end{figure}
\texttt{V} is the type of values, which includes literals, constructors,
closures, and more. The grammar of \texttt{V} is shown in figure~\ref{CakeV}.
\begin{figure}
\begin{alltt}
V
  = LitV Lit
  | ConV (Maybe (ConN, TId_or_Exn)) [V]
  | Closure (Environment V) VarN Exp
  | RecClosure (Environment V) [(VarN, VarN, Exp)] VarN
  | Loc Natural
  | VectorV [V]
\end{alltt}
\caption{Values of CakeML}
\label{CakeV}
\end{figure}
Thus, \texttt{evaluate} takes a list of expressions and applies a number of
semantic rules to yield a list of values. For example, a simple
\texttt{Literal (IntLit 1)} is evaluated to \texttt{LitV (IntLit 1)}.
If something wrong happens, e.g. if a string is given as input to a numeral
operation, an error is returned instead.

More complex expressions require more evaluation, e.g. a \texttt{Let xo e1 e2}
expression requires that the first expression \texttt{e1} is evaluated to a
value and (potentially) stored in the variable \texttt{xo} before evaluating
the second expression \texttt{e2}.

The implemented semantics was unit tested. Simple and common Expressions were
given as input to \texttt{evaluate} and checked to yield the correct result.
A predefined (empty) state and environment was created in order to have a number
of variables and constructors to use in the tests. 
