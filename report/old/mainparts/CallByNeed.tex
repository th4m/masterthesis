\chapter{Optimisation - Implementing Call by Need}
\textit{This chapter is a placeholder for future use, when this has actually been implemented}
This section will present and discuss an optimisation that was implemented in
order to prevent unnecessary and repetitive evaluation of expressions.
With the implementation if laziness, when expressions are evaluated and saved as
variables in the environment, they are in many cases saved as thunks. When the
variables are then used, they are forced (evaluated) in order to yield a proper
value. This happens each time a variable is called.

An optimal usage of a thunk would be to force it once and reuse the yielded
value.

How to achieve this:
\begin{itemize}
\item Save thunks as references. Save reference to subroutine that forces it?
\item When forced, the reference should be changed to point at the value
\end{itemize}

