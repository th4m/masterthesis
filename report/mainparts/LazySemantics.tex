\section{Lazy Semantics}
\label{LazySem}
Lazy semantics was defined for this project.

While the strict semantics was defined as big-step semantics, the lazy semantics
makes use of thunks to make more steps in the evaluation.
This is mainly reflected on the definition of the evaluate function
and the addition of the \texttt{Thunk} constructor to the value (\texttt{V})
type.

\subsection{Thunks}
A new constructor for the value type was added for the lazy semantics, representing
a thunk. A thunk is a type of data-structure that contains an un-evaluated expression
and an environment~\cite{Takano:2015:TRL:2695664.2695693}. This is used for saving
partially evaluated expressions, avoiding the big-step semantics approach that
the strict semantics uses.

\subsection{Evaluate}
\label{Eval}
Instead of recursively calling the \texttt{evaluate} function, as
in the strict semantics, the lazy semantics tries to evaluate as little as possible.
When the lazy \texttt{evaluate} function (called \texttt{evaluateLazy})
is called, it performs the minimum necessary evaluation
for the given expression and returns either an evaluated value or a thunk as a result.
In contrast to the strict semantics, the state is not returned together with the
result. This is because the lazy semantics is pure and does not use any state.

In cases where the fully evaluated value is required, a \texttt{force} function is used.
This function takes an argument of type \texttt{V} (which includes the thunk constructor),
and recursively applies \texttt{evaluate} until a non-thunk value is produced.

%% \vspace{\fboxsep}
%% \begin{minipage}{\linewidth}
\begin{figure}[!ht]
\begin{alltt}
force :: V -> Result [V] V
force (Thunk env e) = case evaluateLazy env [e] of
  RVal [Thunk env' e'] -> force (Thunk env' e')
  res -> res
force v = RVal [v]
\end{alltt}
\end{figure}
%% \end{minipage}
%% \vspace{\fboxsep}

By making use of thunks and step-wise evaluation, the lazy semantics can avoid unnecessary
evaluation. An example of this is \texttt{let} expressions:
\begin{alltt}
  let x = e1 in e2
\end{alltt}
In the strict semantics, \texttt{e1} is fully evaluated before being stored in the
environment under the variable \texttt{x}. 
This evaluation is rather unnecessary in the case where \texttt{x} is never used in
\texttt{e2}. In the lazy semantics, \texttt{e1} is instead wrapped in a thunk together with
the current environment of that step in the evaluation and stored as \texttt{x}. This way,
\texttt{e1} only needs to be evaluated when called in \texttt{e2}.

\subsection{Testing the Lazy Semantics}
\label{LazySemTest}

The lazy semantics was tested in two different methods.
Firstly, lazy evaluation of common
expressions was checked to yield the same results as when using the strict
evaluator (after omitting the state). Secondly, \texttt{evaluateLazy}
was tested to be able to evaluate some expressions that cannot be evaluated
in the strict semantics.


\subsubsection{Comparison}

\texttt{evaluateLazy} should yield the same results as the strict
\texttt{evaluate}. In order to make sure that this was the case, a number of
simple and common expressions were unit tested:

\begin{figure}[!ht]
\begin{alltt}
  compareEval =
    map (force . exLazy) allExps == map (snd . ex) allExps
      where ex     e' = evaluate empty_st ex_env [e']
            exLazy e' = evaluateLazy ex_env [e']
\end{alltt}
\end{figure}

\noindent Here, \texttt{empty\_st} is, as the name implies, an empty state, and
\texttt{ex\_env} is an environment containing some variables and predefined
constructors. \texttt{allExps} is a list of expressions to test.
\texttt{snd}, which takes the second element of a tuple, is applied in order to
omit the state, which was not used in the lazy semantics.


\subsubsection{Laziness}
\label{lazinessSem}
In order to test that \texttt{evaluateLazy} actually does evaluate expressions
in a lazy manner, certain non-terminating expressions were used as part of the
input. While the tested expressions cannot be proven to be non-terminating,
because of the \textit{Halting problem}~\cite{haltingproblem}, they are believed
to be so, as they are constructed to make use of recursion to infinitely loop
into themselves.

Using a \texttt{let} expression as an example:
\begin{alltt}
  let x = e1 in e2
\end{alltt}
If \texttt{e1} happens to be an non-terminating expression, \texttt{evaluate}
would loop infinitely while trying to get the value of \texttt{e1}.
In the case of \texttt{evaluateLazy}, \texttt{e1} is wrapped in a thunk
together with the environment as \texttt{Thunk env e1} before being stored
as \texttt{x}. Thus, evaluation of e1 is delayed until it is called and forced
in \texttt{e2}.
