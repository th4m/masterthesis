\chapter{Lazy Semantics}
\label{LazySem}
Lazy semantics was defined for this project.
While the strict semantics was defined as big-step semantics, the lazy semantics
makes use of thunks to make more steps in the evaluation.
This is mainly reflected on the definition of the evaluate function
and the addition of the \texttt{Thunk} constructor to the value (\texttt{V})
type.

\section{Thunks}
A thunk is a type of data-structure that contains an un-evaluated expression
and an environment~\cite{Takano:2015:TRL:2695664.2695693}.
For the lazy semantics, a new constructor for the value type was added,
representing a thunk, containing an environment and an expression:
\begin{alltt}
Thunk Environment Exp
\end{alltt}
By using thunks, it is possible to take another approach on evaluation of
expressions. Instead of evaluating an expression to e.g. a literal, the
result can be a thunk that contains the expression that has been partially
evaluated, together with information that is required to reach a fully evaluated
value.

With the strict semantics of CakeML, when an expression is to be stored as a
variable, it always needs to evaluated to a value before stored.
As expressions and values are of different types, it is not possible to mix
them. This also means that it is not possible to partially evaluate an
expression to another expression, considering the type definition of the
\texttt{evaluate} function. A thunk can thus not be of the expression type.
By expressing thunks as values, the implemented lazy semantics does not require
any significant change in structure from the original strict semantics. 

\section{Evaluate}
\label{Eval}
Instead of recursively calling the \texttt{evaluate} function, as
in the strict semantics, the lazy semantics tries to evaluate as little as possible.
When the lazy version of \texttt{evaluate} function (called
\texttt{evaluateLazy}) is called, it performs the minimum necessary evaluation
for the given expression and returns either an evaluated value or a thunk as a result.
In contrast to the strict semantics, the state is not returned together with the
result. This is because the lazy semantics is pure and does not use any state.

In cases where the fully evaluated value is required, a \texttt{force} function is used.
This function takes an argument of type \texttt{V} (which includes the thunk constructor),
and recursively applies \texttt{evaluate} until a non-thunk value is produced.

%% \vspace{\fboxsep}
%% \begin{minipage}{\linewidth}
\begin{figure}[!ht]
\begin{alltt}
force :: V -> Result [V] V
force (Thunk env e) = case evaluateLazy env [e] of
  RVal [Thunk env' e'] -> force (Thunk env' e')
  res -> res
force v = RVal [v]
\end{alltt}
\end{figure}
%% \end{minipage}
%% \vspace{\fboxsep}

By making use of thunks and step-wise evaluation, the lazy semantics can avoid unnecessary
evaluation. An example of this is \texttt{let} expressions:
\begin{alltt}
  let x = e1 in e2
\end{alltt}
In the strict semantics, \texttt{e1} is fully evaluated before being stored in the
environment under the variable \texttt{x}. 
This evaluation is rather unnecessary in the case where \texttt{x} is never used in
\texttt{e2}. In the lazy semantics, \texttt{e1} is instead wrapped in a thunk together with
the current environment of that step in the evaluation and stored as \texttt{x}. This way,
\texttt{e1} only needs to be evaluated when called in \texttt{e2}.

Another type expression that was experimented on to draw profits from laziness
was arithmetics. Mainly, multiplication, which has a inductive definition
(Peano's axioms) with a base case stating that
multiplying a factor with 0 will always return 0, was given extra semantics that
checked if the first expression argument was indeed evaluated to 0. If this was
the case, the second argument would simply be thrown away. In the form of an
AST, multiplication has the form:
\begin{alltt}
  App (OPN Times) [e1, e2]
\end{alltt}
Evaluation  ofthis expression begins by force evaluating \texttt{e1} to yield a
pure value \texttt{v1}. If \texttt{v1} is a literal 0, 0 is returned, otherwise
\texttt{e2} is evaluated and multiplied with \texttt{v1}. With this definition
of lazy multiplication, it is possible to have an infinite chain of
multiplication that will terminate if one of the factors is 0.


\section{Testing the Lazy Semantics}
\label{LazySemTest}
The lazy semantics was tested for correctness in two different methods.
Firstly, lazy evaluation of common
expressions was checked to yield the same results as when using the strict
evaluator (after omitting the state). Secondly, \texttt{evaluateLazy}
was tested to be able to evaluate some expressions that normally cannot be
evaluated by using the strict semantics.

\subsection{Comparison}
\texttt{evaluateLazy} should naturally yield the same results as the strict
\texttt{evaluate} when evaluating common types of expressions. In order to
make sure that this was the case, a test
function, called \texttt{compareEval}, was created for unit tests.
\texttt{compareEval} takes a list of expressions and runs a comparison between
the strict \texttt{evaluate} and lazy \texttt{evaluateLazy} functions.
Both functions use the same empty state, called \texttt{empty\_st}, and
an environment containing some variables and predefined constructors,
called \texttt{ex\_env}. In order to omit the state, which is not used in
the lazy semantics, \texttt{snd}, which takes the second element of a tuple,
is applied to the result of the strict \texttt{evaluate}.

\begin{figure}[!ht]
\begin{alltt}
  compareEval expList =
    map (force . exLazy) expList == map (snd . ex) expList
      where ex     e' = evaluate empty_st ex_env [e']
            exLazy e' = evaluateLazy ex_env [e']
\end{alltt}
\end{figure}

Using \texttt{compareEval}, a number of simple and common expressions were
tested. These included expressions such as arithmetics, conditionals, and
string operations. In order to make sure that evaluation would stay correct
throughout the whole development phase, the expressions were collected in a
list that was given as input to \texttt{compareEval} each time something was
changed. As long as the result was \texttt{True}, the evaluator was concluded
to yield correct results.

\subsection{Laziness}
\label{lazinessSem}
In order to test that \texttt{evaluateLazy} actually does evaluate expressions
in a lazy manner, certain non-terminating expressions were used as part of the
input. While the tested expressions cannot be proven to be non-terminating,
because of the \textit{Halting problem}~\cite{haltingproblem}, they are believed
to be so, as they are constructed to make use of recursion to infinitely loop
into themselves.

Using a \texttt{let} expression as an example:
\begin{alltt}
  let x = e1 in e2
\end{alltt}
If \texttt{e1} happens to be an non-terminating expression, \texttt{evaluate}
would loop infinitely while trying to get the value of \texttt{e1}.
In the case of \texttt{evaluateLazy}, \texttt{e1} is wrapped in a thunk
together with the environment as \texttt{Thunk env e1} before being stored
as \texttt{x}. Thus, evaluation of e1 is delayed until it is called and forced
in \texttt{e2}.
