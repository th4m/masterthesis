\section{Lazy Semantics}

Lazy semantics was defined for this project.
When an expression /econtains a sub-expression that affects the 
