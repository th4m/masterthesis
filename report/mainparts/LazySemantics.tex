\section{Lazy Semantics}
\label{LazySem}
Lazy semantics was defined for this project.

While the strict semantics was defined as big-step semantics, the lazy semantics
was created with small-step semantics in mind. This is mainly reflected on the
definition of the evaluate function and the \texttt{Thunk} constructor.

\subsection{Thunks}
A new constructor for the value (\texttt{V}) type was added for the lazy semantics, representing
a thunk. A thunk is a type of data-structure that contains an un-evaluated expression
and an environment\cite{Takano:2015:TRL:2695664.2695693}. This is used for evaluating
expressions in smaller steps.

\subsection{Evaluate}
Instead of recursively calling the \texttt{evaluate} function, as
in the strict semantics, the lazy semantics tries to evaluate as little as possible.
When the lazy \texttt{evaluate} function is called, it performs the minimum necessary evaluation
for the given expression and returns either an evaluated value or a thunk as a result.
In cases where the evaluated value is required, a \texttt{force} function is used.
This function takes an argument of type \texttt{V} (which includes the thunk constructor),
and recursively applies \texttt{evaluate} until a non-thunk value is produced.

By making use of thunks and step-wise evaluation, the lazy semantics can avoid unnecessary
evaluation. An example of this is \texttt{let} expressions:
\begin{lstlisting}[language=haskell]
  let x = e1 in e2
\end{lstlisting}
In the strict semantics, \texttt{e1} is fully evaluated before being stored in the
environment under the variable \texttt{x}. If \texttt{e1} happens to be an non-terminating
expression, \texttt{evaluate} would loop infinitely while trying to get the value of \texttt{e1}.
This evaluation is rather unnecessary in the case where \texttt{x} is never called in
\texttt{e2}. In the lazy semantics, \texttt{e1} is instead wrapped in a thunk together with
the current environment of that step in the evaluation and stored as \texttt{x}. This way,
\texttt{e1} only needs to be evaluated when called in \texttt{e2}.
