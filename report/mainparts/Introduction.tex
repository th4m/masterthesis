\chapter{Introduction}
The main difference between lazy and strict programming languages is the steps
of evaluation that their respective semantics take.
It has been argued that lazy programming (and functional programming in general)
brings improvement to software development in the form of
modularity~\cite{Hu1989}~\cite{Hu2015}.
One example of how laziness can add modularity can be seen when compared to
a strict language: while a strict language always fully evaluates all the
expressions given as parameters, a lazy language only evaluates expressions
when needed~\cite{ThunkHas27:online}.


\section{An Example of Lazy vs Strict Evaluation}
\label{intro:Example}
A useful feature in lazy programming languages is
infinite data structures, such as infinite lists. Using
these infinite data structures will not cause non-termination, as long as the
program is not trying to access all of its elements. In Haskell, a lazy
functional language, it is possible to write \texttt{take 5 [1..]} and get the
result \texttt{[1, 2, 3, 4, 5]}, even though \texttt{[1..]} is
an expression of an infinite list. This is because the list is not generated
until it is needed, and only the necessary number of elements are generated.

Below is an example of how \texttt{take} could be implemented. The function
takes an integer (for the number of elements to take) and a list as arguments.
This assumes that there is a data type definition for \texttt{List} with two
constructors:
\begin{itemize}
  \item \texttt{Cons (a, List as)} for when there is an element \texttt{a} and the rest of the list \texttt{as}
  \item \texttt{Nil} for the empty list
\end{itemize}

\begin{figure}[H]
\begin{alltt}
take n list =
  if n == 0 then
    Nil
  else
    case list of
      Cons (a, as) -> Cons (a, (takeHs n-1 as))
      Nil          -> Nil
\end{alltt}
\end{figure}

\noindent The function can be written like this for any language, but the semantics
changes the strategy of evaluation. E.g. a strict language is likely to evaluate
all of \texttt{list} when pattern matching, while a lazy language is likely to
evaluate only part of the list as much as needed. 


\section{Project Description}

This project will create a compiler that takes code from a lazy programming
language and compiles it into a strict programming language. The resulting
strict code should have the same user-observable evaluation as the given lazy
code. The purpose of creating such a compiler is to explore the formal
connection between lazy and strict evaluation. The compiler will be written in
the programming language Haskell. The source language for the compiler will be
CakeML with custom semantics for laziness and the target language to which
the compiler will output will be CakeML with its standard strict semantics. This
means that the semantics to express laziness will be defined as part of this
project and translated to the original semantics of CakeML.

Internally, the compiler will take a
CakeML function with lazy semantics and translate it to one
or more strict CakeML functions in order to get a semantically equivalent
evaluation. When it comes to handling laziness, lazy functions will be
translated to explicit thunk values~\cite{Ingerman:1961:TWC:366062.366084} in
CakeML. Thunks are expressions that have yet to be evaluated. Thunk values are
not primitive in CakeML. Our compiler will create code that
implements the thunk values in CakeML. The compiler will target the stateful
features of CakeML to avoid repeated evaluation of the same thunk values.

The operational semantics of lazy CakeML will be written as a part of the
project. As the semantics of CakeML is specified in higher-order logic
(HOL)~\cite{HOLInter57:online},
the operational semantics for CakeML will be translated from HOL to Haskell
functions. The compiler will then connect the lazy and strict semantics.

Tests will be written to show that the compiler generates semantically
equivalent code. This project will employ manual unit testing to test
expressions and steps of evaluation.

In summary, the project will consist of three major parts that will be developed
in parallel:
\begin{itemize}
 \item Writing lazy semantics for the programming language CakeML
 \item Creating a compiler that translates the lazy semantics to the original strict semantics of CakeML
 \item Testing the compiler by writing operational semantics in the form of functions and running unit tests.
\end{itemize}
