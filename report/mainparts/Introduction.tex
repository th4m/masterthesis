\section{Introduction}

The main difference between lazy and strict programming languages is the steps
of evaluation that their respective compilers take. While a strict language
always fully evaluates all the expressions given as parameters, a lazy language
only evaluates expressions when needed \cite{ThunkHas27:online}. For example, it is easy to express
infinite data structures, such as infinite lists, in a lazy language. Using
these infinite data structures would not cause non-termination, as long as the
program is not trying to access all of its elements. In Haskell, a lazy
functional language, it is possible to write \texttt{take 5 [1..]} and get the
result \texttt{[1, 2, 3, 4, 5]}, even though \texttt{[1..]} is
an expression of an infinite list. This is because the list is not generated
until it is needed, and only the necessary number of elements are generated.

