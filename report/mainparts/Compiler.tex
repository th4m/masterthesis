\section{The Compiler}

Following the writing and testing of the lazy semantics, the compiler was
created. The compiler takes CakeML expressions and outputs CakeML expressions,
written with lazy semantics in mind. This is performed with a function called
\texttt{compile}.

The representation of thunks is done using the CakeML expression for
constructors. Expressions are wrapped in constructors with either the name
\textit{Thunk} or \textit{Val}, depending on whether they will produce thunks or
non-thunk values. In order for expressions to be saved as thunks, it is also
necessary to wrap them together with the current environment. This is done by
inserting the expression into a \texttt{Fun} expression. A thunk-expression is
thus expressed as \texttt{Con (Just "Thunk") [Fun "" e]} and a value-expression
is expressed as \texttt{Con (Just "Val") [e]}.

Similarly to the lazy semantics, a \texttt{force} function is used when a
non-thunk value is required. While the \texttt{force} function in the lazy
semantics was written in Haskell, this one is written using the CakeML
expressions specified in the abstract syntax tree in Haskell.

%% \vspace{\fboxsep}
%% \begin{minipage}{\linewidth}
\begin{figure}[!ht]
\begin{alltt}
force :: Exp -> Exp
force e =
App OpApp
  [LetRec [("force", "exp"
          , Mat (Var (Short "exp"))
            [(PCon (Just (Short "Thunk")) [PVar "Thunk"]
             , App OpApp [Var (Short "force")
                         , App OpApp [Var (Short "Thunk")
                                     , Literal (IntLit 0)]])
            ,(PCon (Just (Short "Val")) [PVar "Val"]
             , Var (Short "Val"))]
          )] (Var (Short "force"))
  , e]
\end{alltt}
\end{figure}
%% \end{minipage}
%% \vspace{\fboxsep}

\noindent The semantics behind this \texttt{force} is the same as \texttt{force} defined
in the lazy semantics. Pattern matching checks if the given expression is either
a \textit{Thunk} or \textit{Val} constructor. In the case where it is a thunk,
the inner expression is stored as the variable \textit{``Thunk''} and evaluated
by calling and applying it to a dummy argument, followed by another
\texttt{force} on the yielded result. This recursive application of
\texttt{force} occurs until a \textit{Val} constructor is encountered. In this
case, the inner expression is stored as the variable \textit{``Val''} and called
in order to simply return the expression.

With this definition of thunks and values, as well as \texttt{force}, the
compiler emulates the behavior of the lazy semantics. An example is the
\texttt{let} expression that was described in Section \ref{Eval}.

%% \vspace{\fboxsep}
%% \begin{minipage}{\linewidth}
\begin{figure}[!ht]
\begin{alltt}
compile (Let xo e1 e2) =
  Let xo (thunkCompile e1) (thunkCompile e2)
\end{alltt}
\end{figure}
%% \end{minipage}
%% \vspace{\fboxsep}

\noindent Here, both \texttt{e1} and \texttt{e2} are compiled and wrapped with the thunk
constructor. In the case where \texttt{e1} is a non-terminating expression,
this \texttt{let} expression would loop infinitely trying to evaluate
\texttt{e1}. By wrapping \texttt{e1} in a thunk constructor, the
\texttt{evaluate} function will simply create a closure for \texttt{e1} and save
the thunk expression in the environment.

\subsection{Testing the Compiler}

Testing of the compiler was done in two approaches. The first approach was to
control that evaluating the compiled code would not yield a different result
from evaluating the original code. The second approach was to test the laziness
of the compiled code. The tests would check rather simple expressions that do
not terminate when using strict semantics, but do terminate when using lazy
semantics.


\subsubsection{Comparison}

The compiler was tested in a very similar manner as in
section~\ref{LazySemTest.}. In this case, the output of \texttt{evaluate}
was compared between
before and after compiling the input code. Both runs were given the same
dummy environment and empty state. The non-compiled code is run as is, while
the compiled code is forced before being executed.
The test code is as follows:

%% \vspace{\fboxsep}
%% \begin{minipage}{\linewidth}
\begin{figure}[!ht]
\begin{alltt}
compareEval :: Exp -> Bool
compareEval e = strict == lazy
  where strict = ex e
        lazy   = (ex . force . compile) e
        ex e'  = evaluate empty_st ex_env [e']
\end{alltt}
\end{figure}
%% \end{minipage}
%% \vspace{\fboxsep}

\noindent By reusing some functions from the testing of the lazy semantics,
normal expressions were tested with this \texttt{compareEval}
function to confirm that the result of evaluate would not differ between the
non-compiled and compiled code.


\subsubsection{Laziness}

To confirm that the compiler actually does produce code that can be evaluated
lazily using strict semantics, non-terminating expressions were used as part
of the input. These tests were performed similarly to the tests in
section~\ref{lazinessSem}. In this case, however, the strict \texttt{evaluate}
function was used, showing that it is possible to achieve lazy evaluation
using the strict semantics.

Taking the example of the non-terminating expression from section~\ref{Eval}:
\begin{alltt}
  let x = e1 in e2
\end{alltt}
In terms of the AST in Haskell, this expression looks like:
\begin{alltt}
  Let (Just "x") e1 e2
\end{alltt}
Assuming that \texttt{e1} is a non-terminating expression, this \texttt{let}
expression will loop infinitely on the evaluation of \texttt{e1}, never reaching
\texttt{e2}. When used as input to the \texttt{compile} function, the
\texttt{let} expression will be changed to:
\begin{figure}[!ht]
\begin{alltt}
  Let (Just "x") (Con (Just (Short "Thunk")) [Fun "" e1])
                 (Con (Just (Short "Thunk")) [Fun "" e2])
\end{alltt}
\end{figure}

\noindent When evaluated, \texttt{Con (Just (Short "Thunk")) [Fun "" e1]} will
become a value of the \texttt{ConV} constructor with the \texttt{Fun "" e1}
expression becoming a \texttt{Closure} containing the environment and
\texttt{e1}:
\begin{figure}[!ht]
\begin{alltt}
  Let (Just "x")
    (ConV (Just ("Thunk",TypeId "lazy")) [Closure env "" e1])
    (ConV (Just ("Thunk",TypeId "lazy")) [Closure env "" e2])
\end{alltt}
\end{figure}

\noindent As such, the actual evaluation of \texttt{e1} has been avoided
until it is called in \texttt{e2} as the variable \texttt{x}, followed by a
\texttt{force}.

Haskell's \texttt{undefined} was also utilized to test laziness. By definition,
\texttt{undefined} throws an exception when evaluated~\cite{Undefine63:online}.
This means that as long as it is not evaluated, the program will run without
any issues. This was tested in e.g. a \texttt{let} expression:
\begin{alltt}
  Let (Just "x") undefined e2
\end{alltt}
Here, \texttt{undefined} will be wrapped in a thunk constructor and a
\texttt{Fun} expression, just like in the example above:
\begin{alltt}
  Con (Just (Short "Thunk")) [Fun "" undefined]
\end{alltt}
When evaluated, the
\texttt{Fun} expression will simply be turned into a \texttt{Closure} with the
environment and the accompanying expression (in this case \texttt{undefined}):
\begin{figure}[!ht]
\begin{alltt}
  ConV (Just ("Thunk",TypeId (Short "lazy")))
    [Closure env "" undefined]
\end{alltt}
\end{figure}

\noindent This will be saved as the variable \texttt{x}. \texttt{undefined} will
only be evaluated when \texttt{x} is called and forced.
