\chapter{Call-By-Need Semantics}
With the completion of the compiler, expressions bound as variables are stored
as thunks, delaying their evaluation until they are called. While this means
the compiled code exhibits lazy behavior, the thunks are evaluated every
time that the variable is called. The optimal behavior, called
\textit{call-by-need}, is to evaluate the thunks at most once and reuse the
resulting value. This chapter will describe how the compiler was optimised to
use the stateful features of CakeML in order to implement call-by-need
mechanics.

\section{Implementation}
With the implementation of the compiler, as described in
Chapter~\ref{chapter:compiler}, expressions bound by variables have been stored
in the environment as thunks and evaluated when called. In order to use
call-by-need mechanics, the compiler was altered to make use of CakeML's
inherent state in order to save thunks, as well as their evaluated value.

The first and perhaps largest change was that instead of storing variable-bound
expressions in the environment, they were stored in the state. The reference to
the expressions were thus stored in the environment, so that the thunks could be
accessed by using the dereference operator. The second change was made to what
happens when a variable is called, making it call from the state by using the
reference saved in the environment.

In order to be able to utilise the state, there were three operations that were
vital: \texttt{OpRef}, \texttt{OpDeref}, and \texttt{OpAssign}. These have all
been defined under the \texttt{Op} datatype, which is used in combination with
the \texttt{App} expression. When evaluated, \texttt{App OpRef [e]} creates a new
entry in the state for the value that \texttt{e} is evaluated to. The reference
is then returned as a location value \texttt{Loc n}. \texttt{App OpDeref [e]}
takes an expression \texttt{e} that is to be evaluated to a reference (e.g. a
variable that contains the reference), and returns the value that was stored in
the state for the given reference. \texttt{App OpAssign [e1, e2]} takes a
reference \texttt{e1} and an expression \texttt{e2}, and changes the value
stored in the state in the location given by \texttt{e1} to the value that
\texttt{e2} is evaluated to. The result of evaluating \texttt{App OpAssign}
is a a constructor expression containing \texttt{Nothing}, which needs to be
kept in mind when using \texttt{OpAssign}.

The first step of optimising the compiler by using the above operations was to
change the compilation of the \texttt{Let} expression, as it is what binds
expressions to variables. The \texttt{OpRef} operation was used to store the
thunk in the state. The resulting location value was thus stored in the
environment under the variable that originally would be used to store the thunk.
Thus, \texttt{compile} for the \texttt{Let} expression was changed from
\begin{alltt}
  compile (Let xo e1 e2) =
    Let xo (thunkCompile e1) (thunkCompile e2)
\end{alltt}
to
\begin{alltt}
  compile (Let xo e1 e2) =
    Let xo (makeVal (App OpRef [thunkCompile e1])) (thunkCompile e2)
\end{alltt}
With this change, thunks are now stored in the state instead of the environment.
The reference returned from storing the thunk was wrapped in the \texttt{Val}
constructor before being stored in the environment.
What remained was to change what happens when a variable is called.

Calling a variable from the environment is what happens when the \texttt{Var}
expression is evaluated with the strict semantics. The previous implementation
of \texttt{compile} for \texttt{Var} simple returned the variable wrapped in a
thunk:
\begin{alltt}
  compile (Var n) = makeThunk \$ Var n
\end{alltt}
This was changed to take the reference stored in the environment for the
variable and make a call to the state in order to retrieve the stored thunk.
Retrieving the thunk from the state was done with the \texttt{OpDeref}
operation in combination with the reference that the variable \texttt{n} is
pointing at:
\begin{alltt}
  compile (Var n) = makeVal \$ force \$ App OpDeref [force (Var n)]
\end{alltt}
To make the compiler exhibit call-by-need semantics, it was necessary to
change the stored thunk to a value after it was evaluated. This was done by
using the \texttt{OpAssign} operator. The compiler was thus changed to assign
the evaluated thunk to the variable that it was bound to whenever the variable
was called:
\begin{alltt}
  compile (Var n) = assign \$ solve
    where solve  = makeVal \$ force \$ App OpDeref [force (Var n)]
          assign = Let Nothing (App OpAssign [force (Var n), solve])
\end{alltt}
\noindent This solution makes use of the \texttt{Let} expression with \texttt{Nothing} as
variable, as the assignment does not return any particularly useful value. The
\texttt{Let} expression is a partially applied expression, missing the last
expression that should be evaluated after the assignment. This was filled in
with a call to the state to get the desired value.


%% Discussion: use state instead of overshadowing in the environment
%% TODO: fix LetRec to save in the state
%% TODO: fix Let to save a call to the reference instead? Not possible to do it efficiently/reliably? Discuss?
