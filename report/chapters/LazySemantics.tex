\chapter{Lazy Semantics}
The second step of this thesis, after translating the strict semantics of CakeML
to Haskell, was to define lazy semantics that the compiler will be based on.
The lazy semantics defined in Haskell was designed to delay evaluation by
storing partially evaluated expressions in thunks. This chapter will describe
the implementation of the lazy semantics.

\section{Implementation}
%% Modified two modules: semantic primitives and evaluate (mainly) 
%% SemPrim was modified to remove state from certain functions
%% SemPrim was modified to add Thunk to V
%% This definition of lazy evaluation is pure
%% evaluate is redefined in evaluateLazy
%% Doesn't take state
%% Tries to evaluate as little as possible
%% When possible, evaluate takes one (or more) steps in evaluation,
%% by e.g. applying a function from semantic primitives, and then
%% wrapping the resulting expression in a thunk and returning it
%% Implemented force to fully evaluate expressions
%% Attempt 1 was faulty. Could return wrong environment
%% Attempt 2 was successful. Forced certain subexpressions instead.

The task of defining the lazy semantics for CakeML consisted of defining a new
\texttt{evaluate} function (called \texttt{evaluateLazy from now} and
appropriate functions in the semantic primitives.
In order to implement laziness, it was also necessary to incorporate the idea of
thunks, as described in section~\ref{back:lazySem}. As such, thunks were added
as the \texttt{Thunk} constructor in the \texttt{V} datatype. This would
effectively make it a value that could be returned as a result by
\texttt{evaluateLazy}. The main purpose of using thunks was to suspend partially
evaluated expressions within. To further evaluate said expressions, additional
information, contained in the environment, is required.
The \texttt{Thunk} constructor was thus defined as such:

\begin{figure}[H]
\begin{alltt}
  Thunk (Environment V) Exp
\end{alltt}
\end{figure}

\texttt{evaluateLazy}, the lazy version of \texttt{evaluate}, was implemented
with the goal of evaluating expressions as little as possible. While the type
of \texttt{evaluateLazy} needed to stay similar to
\texttt{evaluate}, minor changes were made; mainly, the state was omitted
altogether in order to keep the evaluation pure. The type of
\texttt{evaluateLazy} was thus declared as:

\begin{figure}[H]
\begin{alltt}
  evaluateLazy :: Environment V -> [Exp] -> Result [V] V
\end{alltt}
\end{figure}

With the addition of the \texttt{Thunk} constructor, \texttt{evaluateLazy}
returns either a thunk or any other value. 
The implementation of \texttt{evaluateLazy} was performed in two attempts, which
will be described in the following sections.

\subsection{Attempt 1}


\subsection{Attempt 2}
