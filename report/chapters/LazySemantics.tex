\chapter{Lazy Semantics}
\label{lazySem}
The second step of this thesis, after translating the strict semantics of CakeML
to Haskell, was to define lazy semantics that the compiler will be based on.
The lazy semantics defined in Haskell was designed to delay evaluation by
storing partially evaluated expressions in thunks. This chapter will describe
the implementation of the lazy semantics.

\section{Definition}
\label{lazySem:impl}
%% Modified two modules: semantic primitives and evaluate (mainly) 
%% SemPrim was modified to remove state from certain functions
%% SemPrim was modified to add Thunk to V
%% This definition of lazy evaluation is pure
%% evaluate is redefined in evaluateLazy
%% Doesn't take state
%% Tries to evaluate as little as possible
%% When possible, evaluate takes one (or more) steps in evaluation,
%% by e.g. applying a function from semantic primitives, and then
%% wrapping the resulting expression in a thunk and returning it
%% Implemented force to fully evaluate expressions
%% Attempt 1 was faulty. Could return wrong environment
%% Attempt 2 was successful. Forced certain subexpressions instead.

The task of defining the lazy semantics for CakeML consisted of defining a new
\texttt{evaluate} function (called \texttt{evaluateLazy} from now) and
appropriate functions in the semantic primitives.
In order to implement laziness, it was also necessary to incorporate the idea of
thunks, as described in Section~\ref{back:lazySem}. As such, thunks were added
as the \texttt{Thunk} constructor in the \texttt{V} datatype. This would
effectively make it a value that could be returned as a result by
\texttt{evaluateLazy}. The main purpose of using thunks was to suspend partially
evaluated expressions within. To further evaluate said expressions, additional
information, contained in the environment, is required.
The \texttt{Thunk} constructor was thus added to the \texttt{V} type:

\begin{figure}[H]
\begin{alltt}
  data V =
    ...
    | Thunk (Environment V) Exp
\end{alltt}
\end{figure}

The function \texttt{evaluateLazy}, the lazy version of \texttt{evaluate}, was
implemented with the goal of evaluating expressions as little as possible. While
the type of \texttt{evaluateLazy} needed to stay similar to
\texttt{evaluate}, minor changes were made; mainly, the state was omitted
altogether in order to keep the evaluation pure. The type of
\texttt{evaluateLazy} was thus declared as:

\begin{figure}[H]
\begin{alltt}
  evaluateLazy :: Environment V -> [Exp] -> Result [V] V
\end{alltt}
\end{figure}

\noindent With the addition of the \texttt{Thunk} constructor, \texttt{evaluateLazy}
returns either a thunk or any other value.

Another vital function that needed to be implemented was \texttt{force}.
This function is designed to take a value and pattern match for the
\texttt{Thunk} constructor. When the \texttt{Thunk} case has been entered,
the expression wrapped inside the thunk is extracted and lazily evaluated until
an actual value is yielded. When any other value is caught in the pattern matching,
the value is simply returned as a result. The function \texttt{force} is defined
as such:

\begin{figure}[H]
\begin{alltt}
  force :: V -> Result [V] V
  force (Thunk env e) = case evaluateLazy env [e] of
    RVal [Thunk env' e'] -> force (Thunk env' e')
    res -> res
  force v = RVal [v]
\end{alltt}
\end{figure}

Using the definition of thunks and \texttt{force}, the definition of
\texttt{evaluateLazy} was written in two attempts, which
will be described in the following sections.

\subsection{Attempt 1}
The first (erroneous) attempt of implementing \texttt{evaluateLazy} used a naive
idea of evaluating expressions as little as possible. This was mainly seen in
expressions that contained sub-expressions, e.g. \texttt{If e1 e2 e3}. The first
version of \texttt{evaluateLazy} on \texttt{If} would apply
\texttt{evaluateLazy} on \texttt{e1} and check if it returned a thunk or a
value. If it was a thunk, wrapped together with a partially evaluated expresion
\texttt{e1'} and a potentially different environment \texttt{env'}, it would
create and return a new thunk \texttt{Thunk env' (If e1' e2 e3)}. If it was
another type of value \texttt{v}, it would apply \texttt{do\_if} to yield either
\texttt{e2} or \texttt{e3} as the expression \texttt{e}. This would in turn
, assuming that \texttt{v} was valid, create and return a new thunk containing
\texttt{e} and the environment that was given as input to \texttt{evaluateLazy}.

\begin{figure}[H]
\begin{alltt}
  evaluateLazy st env [If e1 e2 e3]   =
    case evaluateLazy st env [e1] of
      (st', RVal vs)   -> case head vs of
        Thunk env' e1' -> (st', RVal [Thunk env' (If e1' e2 e3)])
        v              -> case do_if v e2 e3 of
          Just e  -> (st', RVal [Thunk env e])
          Nothing -> (st', RErr (RAbort RType_Error))
      res -> res
\end{alltt}
\end{figure}

The problem with this implementation is that the environment is passed along
in each iteration. The strict \texttt{evaluate} evaluates \texttt{e} with
the input environment. This act of passing on an altered environment is an
unwanted process, as \texttt{e1} may assign some variables that were only
intended to be temporary for the evaluation of \texttt{e1}. The strict
\texttt{evaluate} does not exhibit this behavior, but instead keeps the
evaluation of \texttt{e1} in its own scope.

\subsection{Attempt 2}
\label{lazySem:att2}
The second attempt of implementing \texttt{evaluateLazy} was more successful
in creating delayed evaluation without altering more data than wanted. This
was achieved by forcing certain sub-expressions instead of leaving them
suspended. Forcing these sub-expressions produced the value that was needed
to continue the evaluation process. The forcing process was made easier by
creating a function called \texttt{evalAndForce} that applied
\texttt{evaluateLazy} on the head of the given list of expressions and
environment, and forced the resulting value.

\begin{figure}[H]
\begin{alltt}
  evalAndForce :: Environment V -> [Exp] -> Result [V] V
  evalAndForce _env []    = RVal []
  evalAndForce env (e:es) =
    case evaluateLazy env [e] of
      RVal v -> case force (head v) of
        RVal val ->
          case evalAndForce env es of
            RVal vs -> RVal ((head val):vs)
            res -> res
        res -> res
      res -> res
\end{alltt}
\end{figure}

Taking the \texttt{If e1 e2 e3} case as an example again, the condition
\texttt{e1} is forced to produce a (hopefully boolean) value, which is run
through \texttt{do\_if}. The resulting expression, which is the branch that is
to be evaluated in the next step, is wrapped in a thunk and returned as a
result.

\begin{figure}[H]
\begin{alltt}
  evaluateLazy env [If e1 e2 e3]   =
    case evalAndForce env [e1] of
      RVal v ->
        case do_if (head v) e2 e3 of
          Just e  -> RVal [Thunk env e]
          Nothing -> RErr (RAbort RType_Error)
      res -> res
\end{alltt}
\end{figure}

This new idea of forcing sub-expressions was applied to all cases of the
pattern matching in \texttt{evaluateLazy}.

\section{Testing the Definition}
%% Tested the implementation
%% Used the list of expressions that was tested for the strict semantics
%% Created a comparison function that evaluated each expression with both
%% strict and lazy semantics and checked if the results were identical
%% Also tested certain expressions that would end up in bottoms (non-terminating loops)
%% Both test cases were successful

Similarly to the strict semantics, unit testing was performed on the lazy
semantics, using the same dummy environment and expressions. 
The result from the lazy evaluation was compared to what the strict evaluation
would yield to make sure that they were equal. An additional test method was
applied for lazy evaluation as well. This testing consisted of checking for
termination in evaluation of expressions that would normally not terminate with
the strict semantics. These two test methods will be described in this section.

\subsection{Comparing Results}
For comparing results between lazy and strict evaluation, a function, called
\texttt{compareEval} was defined. \texttt{compareEval} takes a list of
expressions and applies \texttt{evaluate} and \texttt{evaluateLazy}
in separate passes. Both evaluation processes use the same dummy environment.
Evaluation with the an empty state (called \texttt{empty\_st}) and
a dummy environment (called \texttt{ex\_env}) was simplified by defining a
function for strict evaluation:

\begin{figure}[H]
\begin{alltt}
  ex e = evaluate empty_st ex_env [e]
\end{alltt}
\end{figure}

\noindent Similarly, evaluation for lazy evaluation with \texttt{ex\_env} was
simplified by defining a function for lazy evaluation:

\begin{figure}[H]
\begin{alltt}
  exForce e = evalAndForce ex_env [e]
\end{alltt}
\end{figure}

\noindent \texttt{evalAndForce} was naturally used, as a pure value is expected
to be compared with the strict evaluation. With the definition of \texttt{ex}
and \texttt{exForce}, \texttt{compareEval} could be defined:

\begin{figure}[H]
\begin{alltt}
  compareEval expList = map exForce expList == map (snd . ex) expList
\end{alltt}
\end{figure}

By collecting a list of expressions, testing was performed by simply running
\texttt{compareEval} on said list. This was done whenever the semantics was
changed and needed to be tested to control that the results would not change.


\subsection{Testing Termination}
As laziness brings the feature of delaying evaluation of expressions, certain
expressions that would lead to non-termination when using the strict semantics
should terminate when using the lazy semantics. For example, when the expression
\texttt{Let x e1 e2} has a non-terminating expression for \texttt{e1}, the
strict semantics will not terminate on the evaluation of \texttt{e1}. However,
the lazy semantics will simply wrap \texttt{e1} in a thunk and bind it to the
variable \texttt{x}, allowing the evaluation to continue to evaluating
\texttt{e2}.

This test method consisted of creating CakeML expressions that would exhibit the
behavior explained above. One such expression was defined and called
\texttt{termExp}. A non-terminating expression \texttt{inf} was also defined to
be used together with \texttt{termExp}.
\texttt{inf} was defined to create a recursive function \texttt{fun}
that simply calls itself to create an infinite loop. \texttt{recExp} used
\texttt{inf} by binding it to a variable \texttt{var} in a \texttt{Let}
expression before returning a string literal "OK".

\begin{figure}[H]
\begin{alltt}
  termExp =
    Let (Just "var") (inf) (Literal (StrLit "OK"))
    where inf =
      LetRec [("fun", "par", App OpApp [Var (Short "fun")
                                       ,Literal (IntLit 0)])]
      (App OpApp [Var (Short "fun"), Literal (IntLit 0)])
\end{alltt}
\end{figure}

\noindent Running \texttt{termExp} with the strict semantics naturally caused a
non-terminating loop, while running \texttt{termExp} with the lazy semantics
resulted with the literal value "OK", as expected.

With the test results showing positive results, the lazy semantics were deemed
to work as intended. This would signal that the groundwork for the compiler
had been completed.
