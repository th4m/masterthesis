\chapter{Introduction}
\label{chapter:intro}
Lazy evaluation is a style of programming where expressions that are bound to variables
are not evaluated until their results are needed in another part of the
program~\cite{Lazyeval6:online}. This is often put in contrast with another
evaluation style called strict evaluation, which always evaluates expressions
immediately as they are to be bound to variables. 

The main difference between lazy and strict programming languages is the steps
of evaluation that their respective semantics take. With strict semantics,
expressions are evaluated when they are created, while with lazy semantics,
expressions are evaluated when they are needed~\cite{ThunkHas27:online}.
This means that in lazy semantics you can give a function an argument
whose evaluation does not terminate, and as long as
that argument is never used, the function itself can still terminate normally.

It has been argued that lazy programming (and functional programming in general)
brings improvement to software development in the form of
modularity~\cite{Hu1989}~\cite{Hu2015}. Laziness allows the creation of certain
types of programs that cannot be executed by using strict semantics.
For example, the \textit{Fibonacci numbers} is an infinite sequence of numbers
where each number is equal to the sum of the two previous numbers.
Defining the list of all Fibonacci numbers as an expression with strict semantics would
cause the execution to never terminate, as it is required for the function to be
fully evaluated when assigning it to a variable. In a lazy context, however,
the expression that defines the list of all Fibonacci numbers would simply not
be evaluted until it is actually needed, meaning that simply assigning it to a
variable would not cause the program to not terminate.

While lazy evaluation can be seen as very beneficial, there are cases where it
would not be optimal. For example, programs where there are
no superfluous expressions would not have much use of lazy evaluation. When it
is known that all expressions bound to variables will be used,
their evaluation is inevitable. The underlying reason why it would be worse
to apply lazy evaluation, rather than strict evaluation, is how lazy evaluation
delays the evaluation of expressions. The act of delaying evaluation creates
an overhead in both memory and processing time.
In cases where there are no superfluous expressions in the code, using lazy
semantics can cause slower evaluation time when compared to
strict semantics. Despite this potential drawback, laziness can be a desirable
feature in many use cases. 

\section{An example of lazy vs strict evaluation}
\label{intro:Example}
In order to gain a better picture of the benefits of using lazy evaluation,
this section will illustrate how lazy and strict evaluation differ by using
an example.

A useful feature in lazy programming languages is
infinite data structures, such as infinite lists. Using
these infinite data structures will not cause non-termination, as long as the
program is not trying to access all of its elements. In Haskell, a lazy
functional language, there exists a function \texttt{take n list} that takes the
first \texttt{n} elements from \texttt{list} and returns a new list with
these elements. It is possible to write \texttt{take 5 [1..]} and get the
result \texttt{[1, 2, 3, 4, 5]}, even though \texttt{[1..]} is
an expression of an infinite list. This is because only the used part of the
list is generated.
Below is an example of how \texttt{take} could be implemented. The function
takes an integer (for the number of elements to take) and a list as arguments.
This assumes that there is a datatype definition for \texttt{List} with two
constructors:
\begin{itemize}
  \item \texttt{Cons (a, List as)} for when there is an element \texttt{a} and the rest of the list \texttt{as}
  \item \texttt{Nil} for the empty list
\end{itemize}

\begin{figure}[H]
\begin{alltt}
  take n list =
    if n == 0 then
      Nil
    else
      case list of
        Cons (a, as) -> Cons (a, (take n-1 as))
        Nil          -> Nil
\end{alltt}
\end{figure}

\noindent The function can be written like this (with some syntax differences)
for any language, but the semantics
changes the strategy of evaluation. For example, a strict language will evaluate
all of \texttt{list} when pattern matching, while a lazy language is likely to
evaluate only part of the list as much as needed. 

The evaluation strategy of strict semantics is restrictive e.g. in the way that
infinite data structures cannot be used. The example above will not terminate
with strict semantics, as the evaluation of the infinite list will end up in an
infinite loop. This thesis will explore the connection between lazy and strict
semantics and how this connection can be used to allow lazy evaluation in a
strict language.
