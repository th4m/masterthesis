\chapter{Lazy to strict compilation}
\label{chapter:compiler}
The third major step of this project is to implement the compiler that takes
lazy CakeML expressions and produces CakeML expressions that retain the lazy
behaviour even when evaluated using the strict semantics.
The compiler is implemented in Haskell by using the ideas
behind the definition of the lazy semantics. This chapter will describe the
implementation of the compiler that compiles CakeML expressions from the lazy
semantics to the strict semantics in such a way that the lazy behaviour
is preserved.

\section{Implementation}
%% Created Compiler.hs
%% Defined compile-function
%% The compile-function takes an expression and returns an expression
%% Defined thunks by using the Con-expression and Fun-expression
%% Defined force-function by using a combination of CakeML expressions
%% The compile-function makes use of thunks and force to delay evaluation
%% A problem with the LetRec expression was that it was not enough to simply compile the inner expressions
%% The force-function expects that variables saved in the environment are wrapped in thunks or vals
%% The function build_rec_env in the strict semantics saves the recursive functions as RecClosures without wrapping
%%   This is an issue, as it's the internal semantics that does this, and it cannot be changed
%%   A workaround was implemented by pulling the RecClosures out of the environment and saving them again, this time wrapped in Val

For the task of creating the compiler, a \texttt{compile} function is defined.
The \texttt{compile} function is designed to take an expression and alter it to
another expression that incorporates the logic of thunks and \texttt{force} defined in
Section~\ref{lazySem:impl}. In order to further explain how this is done,
more information about how thunks and \texttt{force} are defined needs to
be detailed.

\subsection{Thunks as defined in the compiler}
\label{comp:thunk}
In the lazy semantics, thunks are defined as values. As the compiler only makes
use of expressions, values are not available as resources.
The idea is to simulate the behaviour of storing partially evaluated expressions
wrapped inside thunks in the environment as done in the lazy semantics.
For this purpose, it is necessary to store an expression
together with the environment that is in use when the expression is originally
supposed to be evaluated.

Thunks are defined with the use of the \texttt{Fun} expression combined
with the \texttt{Con} expression. When evaluated, \texttt{Fun} expressions
become closures that contain the input expression together with the environment
in use at the time of evaluation. This fulfills the requirements of keeping
the expression together with an environment.

The \texttt{Con} expressions are constructors that allow separation between thunks
and values. Thunks and values are separated with constructor names that
uniquely identify them. For an expression \texttt{e}, thunks are defined and
simplified with the Haskell function \texttt{makeThunk}:

\begin{alltt}
  makeThunk e = Con (Just (Short "Thunk")) [Fun "" e]
\end{alltt}

\noindent As \texttt{e} is put in a \texttt{Fun} expression, it is not
actually evaluated. Instead, it is put inside the resulting closure, awaiting
evaluation when used as an argument to an \texttt{App Op} expression.

A value in the case of the compiler is an expression that does not require
a delayed evaluation. This mainly applies to expressions such as \texttt{Lit},
\texttt{Fun}, and \texttt{Con} that do not require any further evaluation other
than returning the appropriate value. When wrapping
a value around a \texttt{Literal} expression, there
is no need to keep the environment with it, as the evaluation is not to be
delayed. Thus, for an expression \texttt{e}, values are defined and simplified
with the Haskell function \texttt{makeVal}:

\begin{alltt}
  makeVal e = Con (Just (Short "Val")) [e]
\end{alltt}

In order for the constructors \texttt{Thunk} and \texttt{Val} to be usable,
they need to be defined under a type identifier in the environment. Thus,
during the testing phase for the compiler, the constructors are defined under
the type identifier \texttt{lazy}.

\subsection{Defining \texttt{force} in the compiler}
With the definition of thunks and values finished, the task of defining
\texttt{force} remains. In the lazy semantics, \texttt{force} pattern
matches on values to recursively evaluate expressions inside of the input thunks
until they stop producing thunks.

In order to simulate the behaviour of \texttt{force} defined for the lazy
semantics, a number of CakeML expressions are combined:
\texttt{LetRec} for recursion, \texttt{Mat} for pattern
matching, and \texttt{App OpApp} both for applying \texttt{force} recursively
and forcing an evaluation of the closure contained in a thunk.

\begin{alltt}
force :: Exp -> Exp
force e =
  App OpApp [LetRec [("force", "exp"
                     , Mat (Var (Short "exp"))
                       [(PCon (Just (Short "Thunk")) [PVar "Thunk"]
                        , App OpApp [Var (Short "force")
                                    , App OpApp [Var (Short "Thunk")
                                                , Literal (IntLit 0)]])
                       ,(PCon (Just (Short "Val")) [PVar "Val"]
                        , Var (Short "Val"))]
                     )] (Var (Short "force"))
            , e]
\end{alltt}

\noindent This definition of \texttt{force} is based on the Haskell definition
seen in Section~\ref{lazySem:impl}. The abstract syntax shown above is used by
the compiler. The concrete syntax for \texttt{force}, which is easier for the
human eye to read, can be defined as such:
\newpage
\begin{alltt}
  fun force e =
    case e of
      thunk t => force (t 0)
    | val v   => v
\end{alltt}

%Show concrete syntax

\noindent By injecting \texttt{force} into the code, actual values are generated
from thunks. When a \texttt{lazy} constructor is given
as input to \texttt{force}, it is pattern matched to check if it is a
\texttt{Thunk} or a \texttt{Val}. Similarly to the definition of \texttt{force}
for the lazy semantics in Section~\ref{lazySem:impl}, when a \texttt{Thunk}
constructor is given as input, the content of the constuctor is extracted and
forced. In the case of the \texttt{Val} constructor, the content is simply
extracted and returned.

\subsection{The \texttt{compile} function}
With the important elements of thunks and \texttt{force} implemented, the next
step is to utilise them in the \texttt{compile} function. 
As \texttt{compile} takes an expression and returns an expression, the type of
the function is

\begin{alltt}
  compile :: Exp -> Exp
\end{alltt}

\noindent Pattern matching for all expressions in Haskell allows the function to
handle each expression uniquely. With the lazy semantics defined in
Chapter~\ref{lazySem} in mind, \texttt{compile} is implemented with a similar
approach. Expressions are wrapped in either \texttt{Thunk} or
\texttt{Val} constructors. When an expression needs to be fully evaluated,
\texttt{force} is applied.

An example of compiling an expression can be described with the expression
\texttt{If e1 e2 e3}. As seen in Section~\ref{lazySem:att2}, \texttt{e1} needs
to be forced, while \texttt{e2} or \texttt{e3} should be returned as a thunk.
This is handled by applying \texttt{force} on \texttt{e1} after compiling it,
as well as wrapping both \texttt{e2} and \texttt{e3} in thunks after compiling them.
The code is simplified by creating helper functions \texttt{forceCompile},
which is a function composition of \texttt{force} and \texttt{compile}, and
\texttt{thunkCompile}, which is a function composition of thunk wrapping and
\texttt{compile}.

\begin{alltt}
  If (forceCompile e1) (thunkCompile e2) (thunkCompile e3)
\end{alltt}

\noindent When evaluated, this expression takes the following steps:

\begin{enumerate}
\item \texttt{e1} is forced and evaluated to the value \texttt{v}.
\item \texttt{v} is used as argument to \texttt{do\_if} together with \texttt{e2} and \texttt{e3}, both compiled and wrapped in thunks.
\item The resulting expression \texttt{e} is evaluated.
  %% \begin{itemize}
  %%   \item As \texttt{e} is in practice \texttt{Con (Just "Thunk") [Fun "" e]}, the evaluation will return \texttt{ConV (Just ("Thunk",TypeId (Short "lazy"))) [Closure env "" e]}.
  %% \end{itemize}
\end{enumerate}

\noindent In order for the evaluation of the compiled \texttt{If} expression to
yield the same value that the uncompiled version would yield, it simply
needs to be forced before being evaluated.

This logic of wrapping expressions in \texttt{Thunk} and \texttt{Val}
constructors is applied to all cases of \texttt{compile}. Testing (described in
Section~\ref{compiler:test}) showed that all
expressions evaluate correctly, except for one case: the \texttt{LetRec}
expression. This issue (and its solution) will be explained in the following
subsection.

\subsection{Compiling \texttt{LetRec}}
\label{com:letrec}
The expression \texttt{LetRec} is what makes recursive definitions possible.
The expression consists of a list of local function definitions that each have a
function name, argument name, and a function body. This list is followed by
another expression to be evaluated. When a \texttt{LetRec} is evaluated, its
function definitions
are stored in the environment as \texttt{RecClosure} values, each containing
environments that contain all of the function definitions.

Evaluation of \texttt{LetRec} expressions is defined (in Haskell) as such:

\begin{alltt}
  evaluate st env [LetRec funs e] =
    if allDistinct (map (\textbackslash(x,y,z) -> x) funs) then
      evaluate st (env \{v = build_rec_env funs env (v env)\}) [e]
    else
      (st, RErr (RAbort RType_Error))
\end{alltt}

\noindent The problem with the compiler and this evaluation strategy is that it
is the helper function \texttt{build\_rec\_env} that
creates the \texttt{RecClosure} values and stores them in the environment. This
means that the compiler cannot wrap the \texttt{RecClosure} values in the
\texttt{Val} constructor, as the semantics is not a part of the compiler.
\texttt{RecClosure} values are thus naked in the environment, causing \texttt{force}
to result in a type error when called, as it expects the constructors
\texttt{Thunk} and \texttt{Val} in the pattern matching.

In order to create valid \texttt{RecClosure} values in the environment, a
workaround is needed. This workaround
consists of allowing the \texttt{evaluate} function to create the
\texttt{RecClosure} values that are incompatible with \texttt{force} and
redefine them as valid values.
All this is done in the definition of
\texttt{compile} for \texttt{LetRec}:

\begin{alltt}
  compile (LetRec funs e) =
    LetRec (recVals funs) (repl (compile e) (fst3 funs))
    where
      fst3 []           = []
      fst3 ((f,_,_):fs) = (f:fst3 fs)
      repl e []     = e
      repl e (f:fs) = Let (Just f) (makeVal (Var (Short f))) (repl e fs)
      recVals []           = []
      recVals ((f,x,e):xs) =
        (f,x,
         Let (Just f)
         (makeVal (Var (Short f)))
         (compile e)
        ):(recVals xs)
\end{alltt}

\noindent The helper functions are defined as such: \texttt{fst3} takes a list
of tuples of three and creates a list of the first elements of the tuples (this
is used to extract the function identifiers), \texttt{repl} creates new
definitions of the \texttt{RecClosure} values by extracting them from the
environment and storing them after wrapping them with a \texttt{Val}
constructor, \texttt{recVals} goes into each function and creates a \texttt{Val}
constructor for their respective expressions.
The identifiers created in the first step (that creates naked
\texttt{RecClosure} values) are
overshadowed in the environment when they are redefined. The lookup function
for variables searches the variable list in the environment starting from the
head and moves towards the tail, meaning that there will not be any problems
with the wrong \texttt{RecClosure} being used.

\section{Example of evaluation}
\label{com:example}
Evaluating a compiled expression should yield an equivalent result to when
evaluated with the lazy semantics. Similarly to Sections~\ref{strict:example}
and~\ref{lazy:example}, this section will describe how \texttt{take}
is evaluated after being compiled. The compiled expression is assumed to be
forced at top level before being evaluated.

The evaluation of the compiled \texttt{LetRec} will yield a \texttt{RecClosure}.
With the workaround described in Section~\ref{com:letrec}, the \texttt{RecClosure}
will be properly wrapped with a \texttt{Val} constructor for proper forcing.
When applied with an integer literal, the evaluation of the \texttt{RecClosure}
will return an evaluated \texttt{Fun} expression as a \texttt{Closure} wrapped
in a \texttt{Val} constructor. When forced, the \texttt{Closure} can be applied
to a list \texttt{ls} by using \texttt{App OpApp}. For the evaluation of the
compiled \texttt{If} expression, as the condition is forced, it will yield a
usable value. The two branches are wrapped in thunks, meaning that the result
of \texttt{do\_if} will yield a thunk. In the case where the condition is true
(\texttt{n} is equal to 0), a \texttt{nil} constructor is simply returned.
In the case where the condition is false, the list \texttt{ls} is pattern matched
to check its constructor. While the compilation of the \texttt{Mat} expression
will apply a force on \texttt{ls}, the evaluation of \texttt{Con} does not force
the arguments, as described in Section~\ref{lazy:example}. Thus, a new
definition of \texttt{force} is needed, similar to the one defined for the lazy
semantics. This separate \texttt{force} is used after evaluation to yield a
fully evaluated list.


\section{Testing the compiler}
\label{compiler:test}
The compiler is tested similarly to the lazy semantics. Firstly, a number of
common expressions are compiled and evaluated, followed by a comparison with
the evaluation of the non-compiled expressions. Secondly, termination is tested
to see if expressions that do not terminate when evaluated with the strict
semantics do terminate after compiling and evaluating them with the same
strict semantics.

The evaluation of non-compiled expressions uses the function \texttt{ex},
defined in Section~\ref{lazy:compres}, that uses an empty state and
dummy environment, as when testing the lazy semantics. Evaluation of
compiled expressions is simplified with a function called \texttt{efc} that
compiles, forces, and evaluates a given expression:

\begin{alltt}
  efc = ex . force . compile
\end{alltt}

\noindent The two evaluations \texttt{ex} and \texttt{efc} are then compared
with a function called \texttt{compareEval}:
\newpage
\begin{alltt}
  compareEval :: Exp -> Bool
  compareEval e = strict == lazy
    where strict = ex e
          lazy   = efc e
\end{alltt}

\noindent
With the results of \texttt{compareEval} giving a \texttt{True} value when
tested with a number of expressions, the compiler is deemed to work as intended.

Termination is tested for the compiler by running \texttt{efc} for expressions
that do not terminate when simply evaluated with the strict semantics, but do
terminate with the lazy semantics. For example, the expression defined for the
testing of the lazy semantics:

\begin{alltt}
  termExp =
    Let (Just "var") (inf) (Literal (StrLit "OK"))
    where inf =
      LetRec [("fun", "par", App OpApp [Var (Short "fun")
                                       ,Literal (IntLit 0)])]
      (App OpApp [Var (Short "fun"), Literal (IntLit 0)])
\end{alltt}

\noindent When executed with \texttt{efc}, this expression gives the literal
value "OK", which indicates that the compiler indeed produces expressions that
exhibit laziness.

While the compiler at this stage successfully delays expressions in thunks
until they are called, it is still not quite producing truly lazy expressions.
True laziness is mostly associated with call-by-need semantics, where thunks
retain the values that their expressions are evaluated to~\cite{pltbook}.
The semantics that
the compiler is exhibiting at this stage is called \textit{call-by-name}, where
evaluations are simply delayed. If the thunk is used more than once, it will
be evaluated each time it is used~\cite{DragonBook}.
Call-by-need semantics can be seen as more beneficial than call-by-name, as
repeated use of e.g. variables would require repeated use of thunks.
The next stage of this project is thus to optimise the compiler by implementing
call-by-need semantics.
