\chapter{Compiler}
\label{chapter:compiler}
The third major step of this project was to implement the compiler that takes
regular strict CakeML expressions and produces CakeML expressions that allows
lazy evaluation. The compiler has been implemented in Haskell by using the ideas
behind the definition of the lazy semantics. This chapter will describe the
implementation of the compiler that compiles CakeML expressions from the lazy
semantics into the strict semantics in such a way that the lazy behaviour
is preserved.

\section{Implementation}
%% Created Compiler.hs
%% Defined compile-function
%% The compile-function takes an expression and returns an expression
%% Defined thunks by using the Con-expression and Fun-expression
%% Defined force-function by using a combination of CakeML expressions
%% The compile-function makes use of thunks and force to delay evaluation
%% A problem with the LetRec expression was that it was not enough to simply compile the inner expressions
%% The force-function expects that variables saved in the environment are wrapped in thunks or vals
%% The function build_rec_env in the strict semantics saves the recursive functions as RecClosures without wrapping
%%   This is an issue, as it's the internal semantics that does this, and it cannot be changed
%%   A workaround was implemented by pulling the RecClosures out of the environment and saving them again, this time wrapped in Val

For the task of creating the compiler, a \texttt{compile} function was defined.
The \texttt{compile} function was designed to take an expression and alter it to
another expression that incorporates the logic of thunks and \texttt{force} defined in
Section~\ref{lazySem:impl}. In order to further explain how this was done,
more information about how thunks and \texttt{force} were implemented needs to
be detailed.

\subsection{Thunks as defined in the compiler}
\label{comp:thunk}
In the lazy semantics, thunks were defined as values. As the compiler only makes
use of expressions, values are not available as resources.
The idea is to simulate the behaviour of storing partially evaluated expressions
wrapped inside thunks in the environment as done in the lazy semantics.
For this purpose, it is necessary to store an expression
together with the environment that is in use when the expression is originally
supposed to be evaluated.

Thunks were defined with the use of the \texttt{Fun} expression combined
with the \texttt{Con} expression. When evaluated, \texttt{Fun} expressions
become closures that contain the input expression together with the environment
in use at the time of evaluation. This fulfills the requirements of keeping
the expression together with an environment.

\texttt{Con} expressions are constructors that allow separation between thunks
and values. Thunks and values are separated with constructor names that
uniquely identified them. For an expression \texttt{e}, thunks were defined and
simplified with the Haskell function \texttt{makeThunk}:

\begin{figure}[H]
\begin{alltt}
  makeThunk e = Con (Just (Short "Thunk")) [Fun "" e]
\end{alltt}
\end{figure}

\noindent As \texttt{e} is put in a the \texttt{Fun} expression, it is not
actually evaluated. Instead, it is put inside the resulting closure, awaiting
evaluation when called as an argument to an \texttt{App Op} expression.

A value in the case of the compiler is an expression that does not require
a delayed evaluation. This mainly applies to expressions such as \texttt{Lit},
\texttt{Fun}, and Con that do not require any further evaluation other than returning
the appropriate value. When wrapping
a value around a \texttt{Con} expression, there
is no need to keep the environment with it, as the evaluation is not to be
delayed. Thus, for an expression \texttt{e}, values were defined and simplified
with the Haskell function \texttt{makeVal}:

\begin{figure}[H]
\begin{alltt}
  makeVal e = Con (Just (Short "Val")) [e]
\end{alltt}
\end{figure}

In order for the constructors \texttt{Thunk} and \texttt{Val} to be usable,
they need to be defined under a type identifier in the environment. Thus,
when the compiler was tested, the constructors were defined under the type
identifier \texttt{lazy}.

\subsection{Defining \texttt{force} in the compiler}
With the definition of thunks and values finished, the task of defining
\texttt{force} remained. In the lazy semantics, \texttt{force} pattern
matched on values to recursively evaluate expressions inside of the thunks until
they stopped producing thunks.

In order to simulate the behaviour of \texttt{force} defined for the lazy
semantics, a number of CakeML expressions were combined:
\texttt{LetRec} for recursion, \texttt{Mat} for pattern
matching, and \texttt{App OpApp} both for applying \texttt{force} recursively
and forcing an evaluation of the closure contained in a thunk.

\begin{figure}[H]
\begin{alltt}
force :: Exp -> Exp
force e =
  App OpApp [LetRec [("force", "exp"
                     , Mat (Var (Short "exp"))
                       [(PCon (Just (Short "Thunk")) [PVar "Thunk"]
                        , App OpApp [Var (Short "force")
                                    , App OpApp [Var (Short "Thunk")
                                                , Literal (IntLit 0)]])
                       ,(PCon (Just (Short "Val")) [PVar "Val"]
                        , Var (Short "Val"))]
                     )] (Var (Short "force"))
            , e]
\end{alltt}
\end{figure}

\noindent This definition of \texttt{force} was injected into the code in order
to generate actual values from thunks. When a \texttt{lazy} constructor is given
as input to \texttt{force}, it will be pattern matched to check if it is a
\texttt{Thunk} or a \texttt{Val}. Similarly to the definition of \texttt{force}
for the lazy semantics in Section~\ref{lazySem:impl}, when a \texttt{Thunk}
constructor is given as input, the content of the constuctor is extracted and
forced. In the case of the \texttt{Val} constructor, the content is simply
extracted and returned.

\subsection{The \texttt{compile} function}
With the important elements of thunks and \texttt{force} implemented, the next
step was to utilise them in the \texttt{compile} function. 
As \texttt{compile} takes an expression and returns an expression,the type of
the function is

\begin{figure}[H]
\begin{alltt}
  compile :: Exp -> Exp
\end{alltt}
\end{figure}

\noindent Pattern matching for all expressions in Haskell allows the function to
handle each expression uniquely. With the lazy semantics defined in
Chapter~\ref{lazySem} in mind, \texttt{compile} was implemented with a similar
approach. Expressions were wrapped in either \texttt{Thunk} or
\texttt{Val} constructors. When an expression needs to be fully evaluated,
\texttt{force} is applied.

An example of compiling an expression can be described with the expression
\texttt{If e1 e2 e3}. As seen in Section~\ref{lazySem:att2}, \texttt{e1} needs
to be forced, while \texttt{e2} or \texttt{e3} should be returned as a thunk.
This is handled by applying \texttt{force} on \texttt{e1} after compiling it,
as well as wrapping \texttt{e2} and \texttt{e3} after compiling them.
The code was simplified by creating helper functions \texttt{forceCompile},
which is a function composition of \texttt{force} and \texttt{compile}, and
\texttt{thunkCompile}, which is a function composition of thunk wrapping and
\texttt{compile}.

\begin{figure}[H]
\begin{alltt}
  If (forceCompile e1) (thunkCompile e2) (thunkCompile e3)
\end{alltt}
\end{figure}

\noindent When evaluated, this expression takes the following steps:

\begin{enumerate}
\item \texttt{e1} is forced and evaluated to the value \texttt{v}.
\item \texttt{v} is used as argument to \texttt{do\_if} together with \texttt{e2} and \texttt{e3}, both compiled and wrapped in thunks.
\item The resulting expression \texttt{e} is evaluated.
  %% \begin{itemize}
  %%   \item As \texttt{e} is in practice \texttt{Con (Just "Thunk") [Fun "" e]}, the evaluation will return \texttt{ConV (Just ("Thunk",TypeId (Short "lazy"))) [Closure env "" e]}.
  %% \end{itemize}
\end{enumerate}

\noindent In order for the evaluation of the compiled \texttt{If} expression to
yield the same value that the uncompiled version would yield, it would simply
need to be forced before being evaluated.

This logic of wrapping expressions in \texttt{Thunk} and \texttt{Val}
constructors was applied to all cases of \texttt{compile}. When tested, all
expressions were evaluated correctly, except for one case: the \texttt{LetRec}
expression. This issue (and its solution) will be explained in the following
subsection.

\subsection{Compiling \texttt{LetRec}}
The expression \texttt{LetRec} is what makes recursive definitions possible.
The expression consists of a list of local function definitions that each have a
function name, argument name, and a function body. This list is followed by
another expression to be evaluated. When evaluated, these function definitions
are stored in the environment as \texttt{RecClosure} values, each containing
environments that contain all of the function definitions.

Evaluation of \texttt{LetRec} expressions is defined as such:

\begin{figure}[H]
\begin{alltt}
  evaluate st env [LetRec funs e] =
    if allDistinct (map (\textbackslash(x,y,z) -> x) funs) then
      evaluate st (env \{v = build_rec_env funs env (v env)\}) [e]
    else
      (st, RErr (RAbort RType_Error))
\end{alltt}
\end{figure}

\noindent The problem with the compiler and this evaluation strategy is that it
is the function \texttt{build\_rec\_env} from the semantic primitives that
creates the \texttt{RecClosure}s and stores them in the environment. This means
that the compiler cannot wrap the \texttt{RecClosure} values in the
\texttt{Val} constructor, as the semantics is not a part of the compiler.
\texttt{RecClosure}s are thus naked in the environment, causing \texttt{force}
to result in a type error when called, as it pattern matches on the constructors
\texttt{Thunk} and \texttt{Val}.

A workaround was created to
make valid \texttt{RecClosure} values in the environment. This workaround
consisted of allowing the \texttt{evaluate} function to create the
\texttt{RecClosure} values that were incompatible with \texttt{force} and
reassigning them as valid values. In order to not create possible conflicts
with variable names, the first time that the \texttt{RecClosure} values are
stored in the environment, they are given invalid names (by appending a space
to them). When they are reassigned as properly wrapped in constructors, they
are given their true names. All this was done in the definition of
\texttt{compile} for \texttt{LetRec}:

\begin{figure}[H]
\begin{alltt}
  compile (LetRec funs e) =
    LetRec (recVals funs) (repl (compile e) (fst3 funs))
    where
      fst3 []           = []
      fst3 ((f,_,_):fs) = (f:fst3 fs)
      repl e []     = e
      repl e (f:fs) = Let (Just f) (makeVal (Var (Short f))) (repl e fs)
      recVals []           = []
      recVals ((f,x,e):xs) =
        (f,x,
         Let (Just f)
         (makeVal (Var (Short f)))
         (compile e)
        ):(recVals xs)
\end{alltt}
\end{figure}

\section{Testing the compiler}
\label{compiler:test}
The compiler was tested similarly to the lazy semantics. Firstly, a number of
common expressions were compiled and evaluated, followed by a comparison with
the evaluation of the non-compiled expressions. Secondly, termination was tested
to see if expressions that would not terminate when evaluated with the strict
semantics would terminate after compiling and evaluating them with the same
strict semantics.

The evaluation of non-compiled expressions used the function \texttt{ex},
defined in Section~\ref{lazy:compres}, that uses an empty state and
dummy environment, as when testing the lazy semantics. Evaluation of
compiled expressions was simplified with a function called \texttt{efc} that
compiles, forces, and evaluates a given expression:

\begin{figure}[H]
\begin{alltt}
  efc = ex . force . compile
\end{alltt}
\end{figure}

\noindent The two evaluations \texttt{ex} and \texttt{efc} were then compared
with a function called \texttt{compareEval}:

\begin{figure}[H]
\begin{alltt}
  compareEval :: Exp -> Bool
  compareEval e = strict == lazy
    where strict = ex e
          lazy   = efc e
\end{alltt}
\end{figure}

\noindent As the results of \texttt{compareEval} gave a \texttt{True} value,
the compiler was deemed to work as intended.

Termination was tested for the compiler by running \texttt{efc} for expressions
that do not terminate when simply evaluated with the strict semantics, but would
terminate with the lazy semantics. For example, the expression used in the lazy
semantics:

\begin{figure}[H]
\begin{alltt}
  termExp =
    Let (Just "var") (inf) (Literal (StrLit "OK"))
    where inf =
      LetRec [("fun", "par", App OpApp [Var (Short "fun")
                                       ,Literal (IntLit 0)])]
      (App OpApp [Var (Short "fun"), Literal (IntLit 0)])
\end{alltt}
\end{figure}

\noindent When executed with \texttt{efc}, this expression gave the literal
value "OK", which indicated that the compiler indeed produces expressions that
exhibit lazy traits.

While the compiler at this stage successfully delays expressions in thunks
until they are called, it is still not quite producing truly lazy expressions.
True laziness is often associated with call-by-need semantics, where thunks
retain the values that their expressions were evaluated to~\cite{pltbook}
The semantics that
the compiler is exhibiting at this stage is called \texttt{call-by-name}, where
evaluations are simply delayed. If the thunk is used more than once, it will
be evaluated each time it is called~\cite{DragonBook}.
Call-by-need semantics can be seen as more beneficial than call-by-name, as
repeated use of e.g. variables would require repeated use of thunks.
The next stage of this project was thus to optimise the compiler by implementing
call-by-need semantics.
