\chapter{Compiler}
\label{chapter:compiler}
The third major step of this thesis was to implement the compiler that takes
regular strict CakeML expressions and produces CakeML expressions that allows
lazy evaluation. The compiler has been defined in Haskell by using the ideas
behind the definition of the lazy semantics. However, as it both takes and
produces CakeML expressions, the compiler is run with the premise of using the
original strict semantics. This chapter will describe the implementation of the
compiler that makes CakeML expressions lazy.

\section{Implementation}
%% Created Compiler.hs
%% Defined compile-function
%% The compile-function takes an expression and returns an expression
%% Defined thunks by using the Con-expression and Fun-expression
%% Defined force-function by using a combination of CakeML expressions
%% The compile-function makes use of thunks and force to delay evaluation
%% A problem with the LetRec expression was that it was not enough to simply compile the inner expressions
%% The force-function expects that variables saved in the environment are wrapped in thunks or vals
%% The function build_rec_env in the strict semantics saves the recursive functions as RecClosures without wrapping
%%   This is an issue, as it's the internal semantics that does this, and it cannot be changed
%%   A workaround was implemented by pulling the RecClosures out of the environment and saving them again, this time wrapped in Val

For the task of creating the compiler, a \texttt{compile} function was defined.
\texttt{compile} was designed to take an expression and alter it to another
expression that incorporates the logic of thunks and \texttt{force} defined in
section~\ref{lazySem:impl}. In order to further explain how this was done,
more information about how thunks and \texttt{force} were implemented needs to
be detailed.

\section{Thunks in the Compiler}
\label{comp:thunk}
In the lazy semantics, thunks were defined as values. As the compiler only makes
use of expressions, values are not available as resources.
The idea is to simulate the behavior of storing
partially evaluated expressions wrapped inside thunks in the environment as
done in the lazy semantics. To do that it is necessary to store an expression
together with the environment that is in use when the expression is originally
supposed to be evaluated.

Thunks were defined with the use of the \texttt{Fun} expression combined
with the \texttt{Con} expression. When evaluated, \texttt{Fun} expressions
become closures that contain the input expression together with the environment
in use at the time of evaluation. This fulfills the requirements of keeping
the expression together with an environment.

\texttt{Con} expressions are constructors that allow separation between thunks
and values. Thunks and values were separated with constructor names that
uniquely identified them. For an expression \texttt{e}, thunks were defined as:

\begin{figure}[H]
\begin{alltt}
  Con (Just (Short "Thunk")) [Fun "" e]
\end{alltt}
\end{figure}

\noindent As \texttt{e} is put in a the \texttt{Fun} expression, it is not
actually evaluated. Instead, it is put inside the resulting closure, awaiting
evaluation when called as an argument to an \texttt{App Op} expression.

A value in the case of the compiler is an expression that does not require
a delayed evaluation. This mainly applies to expressions such as \texttt{Lit}
and \texttt{Fun}. When wrapping a value around a \texttt{Con} expression, there
is no need to keep the environment with it, as the evaluation is not to be
delayed. Thus, for an expression \texttt{e}, values were defined as:

\begin{figure}[H]
\begin{alltt}
  Con (Just (Short "Val")) [e]
\end{alltt}
\end{figure}

In order for the constructors "\textit{Thunk}" and "\textit{Val}" to be usable,
they need to be defined under a type identifier in the environment. Thus,
when the compiler was tested, the constructors were defined under the type
identifier "\textit{lazy}".

\section{\texttt{force} in the Compiler}
With the definition of thunks and values finished, the task of defining
\texttt{force} remained. In the lazy semantics, \texttt{force} pattern
matched on values to recursively evaluate expressions inside of the thunks until
they stopped producing thunks. As the compiler does not perform any evaluation
itself, it needs to somehow make the actual semantics make this pattern matching
and recursive evaluation. Thus, \texttt{force} was defined by combining a number
of CakeML expressions: \texttt{LetRec} for recursion, \texttt{Mat} for pattern
matching, and \texttt{App OpApp} both for applying \texttt{force} recursively
and forcing an evaluation.

\begin{figure}[H]
\begin{alltt}
force :: Exp -> Exp
force e =
  App OpApp [LetRec [("force", "exp"
                     , Mat (Var (Short "exp"))
                       [(thunkPat [PVar "Thunk"]
                        , App OpApp [Var (Short "force")
                                    , App OpApp [Var (Short "Thunk")
                                                , Literal (IntLit 0)]])
                       ,(valPat [PVar "Val"]
                        , Var (Short "Val"))]
                     )] (Var (Short "force"))
            , e]
\end{alltt}
\end{figure}


\section{The \texttt{compile} function}
With the important elements of thunks and \texttt{force} implemented, the next
step was to utilise them in the \texttt{compile} function. 
\texttt{compile} takes an expression and returns an expression. Thus the type of
the function is

\begin{figure}[H]
\begin{alltt}
  compile :: Exp -> Exp
\end{alltt}
\end{figure}

\noindent Pattern matching for all expressions in Haskell allows the function to
handle each expression uniquely. With the lazy semantics defined in
chapter~\ref{lazySem} in mind, \texttt{compile} was implemented with a similar
approach. Expressions were wrapped in either "\textit{Thunk}" of
"\textit{Value}" constructors. When an expression needs to be fully evaluated,
\texttt{force} is applied.

An example of compiling an expression can be described with the expression
\texttt{If e1 e2 e3}. As seen in section~\ref{lazySem:att2}, \texttt{e1} needs
to be forced, while \texttt{e2} or \texttt{e3} should be returned as a thunk.
This is handled by applying \texttt{force} on \texttt{e1} after compiling it,
as well as wrapping \texttt{e2} and \texttt{e3} after compiling them.
The code was simplified by creating helper functions \texttt{forceCompile},
which is a function composition of \texttt{force} and \texttt{compile}, and
\texttt{thunkCompile}, which is a function composition of thunk wrapping and
\texttt{compile}.

\begin{figure}[H]
\begin{alltt}
  If (forceCompile e1) (thunkCompile e2) (thunkCompile e3)
\end{alltt}
\end{figure}

\noindent When evaluated, this expression takes the following steps:

\begin{enumerate}
\item e1 is forced and evaluated to the value \texttt{v}
\item \texttt{v} is used as argument to \texttt{do\_if} together with \texttt{e2} and \texttt{e3}, both compiled and wrapped in thunks.
\item The resulting expression \texttt{e} is evaluated.
  %% \begin{itemize}
  %%   \item As \texttt{e} is in practice \texttt{Con (Just "Thunk") [Fun "" e]}, the evaluation will return \texttt{ConV (Just ("Thunk",TypeId (Short "lazy"))) [Closure env "" e]}.
  %% \end{itemize}
\end{enumerate}

\noindent In order for the evaluation of the compiled \texttt{If} expression to
yield the same value that the uncompiled version would yield, it would simply
need to be forced before being evaluated.

This logic of wrapping expressions in "\textit{Thunk}" and "\textit{Val}"
constructors was applied to all cases of \texttt{compile}. When tested, all
expressions were evaluated correctly, except for one case: the \texttt{LetRec}
expression. This issue (and its solution) will be explained in the following
subsection.

\subsection{Compiling \texttt{LetRec}}
\texttt{LetRec} is the expression that makes recursive definitions possible.
The expression consists of a list of local function definitions that each have a
function name, argument name, and a function body. This list is followed by
another expression to be evaluated. When evaluated, these function definitions
are stored in the environment as \texttt{RecClosure} values, each containing
environments that contain all of the function definitions.

\texttt{evaluate} for \texttt{LetRec} is defined as such:

\begin{figure}[H]
\begin{alltt}
  evaluate st env [LetRec funs e] =
    if allDistinct (map (\\(x,y,z) -> x) funs) then
      evaluate st (env {v = build_rec_env funs env (v env)}) [e]
    else
      (st, RErr (RAbort RType_Error))
\end{alltt}
\end{figure}

\noindent The problem with the compiler and this evaluation strategy is that it
is the function \texttt{build\_rec\_env} from the semantic primitives that
creates the \texttt{RecClosure}s and stores them in the environment. This means
that the compiler cannot wrap the \texttt{RecClosure} values in the
"\textit{Val}" constructor, as the semantics is not a part of the compiler.
\texttt{RecClosure}s are thus naked in the environment, causing \texttt{force}
to result in a type error when called, as it pattern matches on the constructors
"\textit{Thunk}" and "\textit{Val}".

As the evaluation of \texttt{LetRec} expressions will cause calls to the locally
defined functions to always result in type errors, a workaround was created to
make valid\texttt{RecClosure} values in the environment. This workaround
consisted of allowing the \texttt{evaluate} function to create the
\texttt{RecClosure} values that were incompatible with \texttt{force} and
reassigning them as valid values. In order to not create possible conflicts
with variable names, the first time that the \texttt{RecClosure} values are
stored in the environment, they are given invalid names (by appending a space
to them). When they are reassigned as properly wrapped in constructors, they
are given their true names. All this was done in the definition of
\texttt{compile} for \texttt{LetRec}:

\begin{figure}[H]
\begin{alltt}
compile (LetRec funs e) =
  LetRec (replace funs invN) (reinstance invN names (thunkCompile e))
  where names = getNames funs
        getNames [] = []
        getNames ((f,_,_):funs') = (f:getNames funs')
        invN = map (++ " ") names
\end{alltt}
\end{figure}

\section{Testing the Implementation}
