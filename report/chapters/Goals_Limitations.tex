\chapter{Goals and Limitations}
This chapter will describe the goals and limitations of this thesis.

\section{Project Description}
\label{intro:project}
The goal of this project is to create a compiler that takes code from a strict
programming language and compiles it to allow for a lazy evaluation style. The
resulting code should have the same user-observable evaluation as the given
strict code. The purpose of creating such a compiler is to explore the formal
connection between lazy and strict evaluation. The compiler will be written in
the programming language Haskell. The source language for the compiler will be
CakeML (which has strict semantics ) and the target language to which the
compiler will output will be CakeML, but with support for lazy evaluation. This
means that the semantics to express laziness will be defined as part of this
project and used as a base to implement the compiler.

Internally, the compiler will take a
CakeML function with lazy semantics and translate it to one
or more strict CakeML functions in order to get a semantically equivalent
evaluation. When it comes to handling laziness, lazy functions will be
translated to explicit thunk values~\cite{Ingerman:1961:TWC:366062.366084} in
CakeML. Thunks are expressions that have yet to be evaluated. Thunk values are
not primitive in CakeML. Our compiler will create code that
implements the thunk values in CakeML. The compiler will target the stateful
features of CakeML to avoid repeated evaluation of the same thunk values.

The operational semantics of lazy CakeML will be written as a part of the
project. As the semantics of CakeML is specified in higher-order logic
(HOL)~\cite{HOLInter57:online},
the operational semantics for CakeML will be translated from HOL to Haskell
functions. The compiler will then connect the lazy and strict semantics.

Tests will be written to show that the compiler generates semantically
equivalent code. This project will employ manual unit testing to test
expressions and steps of evaluation.

In summary, the project will consist of three major parts that will be developed
in parallel:
\begin{itemize}
 \item Writing lazy semantics for the programming language CakeML
 \item Creating a compiler that translates the lazy semantics to the original strict semantics of CakeML
 \item Testing the compiler by writing operational semantics in the form of functions and running unit tests.
\end{itemize}

\section{Limitations}
Due to the time constraints of this thesis, certain limitations will be enforced
to keep the workload within a reasonable size. These limitations are:

\begin{itemize}
\item Expressions will be the highest level of the grammar that will be implemented
\item No parser will be written to take actual CakeML code in string format
\item Only the version of CakeML at the start of this thesis will be used as base for the implementation of the compiler
\end{itemize}

As expressions make up of the majority of the language, they will be the highest
category of the grammar that will be implemented. This means that declarations
will not be a part of this thesis.

No actual parser that takes written CakeML code will be created for this thesis.
Instead, an internal representation of CakeML expressions will be created in
Haskell and be used to create the semantics and compiler. A parser that
takes CakeML code in string format and converts it to the internal
representation can be created at any time. It should not prove to be a
challenging task to link said parser together with the compiler.

At the time of writing, the programming language CakeML is still evolving at a
high pace, meaning that the semantics of the language is prone to change at
any time. In order to keep a steady workflow, any updates to the semantics after
the start of this thesis will not be considered.
