\chapter{Goals and Limitations}
This chapter will describe the goals and limitations of this thesis.

\section{Project Description}
\label{intro:project}
The goal of this project is to create a compiler that takes code from a lazy
programming language and compiles it to a target language that has strict
evaluation. The resulting code should have the same user-observable evaluation
as the given lazy code.
The purpose of creating such a compiler is to explore the formal
connection between lazy and strict evaluation. The compiler will be written in
the programming language Haskell. The source language for the compiler will be
a lazy definition of CakeML's semantics and the target language to which the
compiler will output will be CakeML, with its original strict semantics. This
means that the semantics to express laziness will be defined as part of this
project and used as a base to implement the compiler.

Internally, the compiler will take a
CakeML expression with lazy semantics and translate it to one
or more strict CakeML expressions in order to get a semantically equivalent
evaluation. When it comes to handling laziness, lazy expressions will be
translated to explicit thunk values~\cite{Ingerman:1961:TWC:366062.366084} in
CakeML. Thunks are expressions that have yet to be evaluated. Thunk values are
not primitive in CakeML. Our compiler will create code that
implements the thunk values in CakeML. The compiler will target the stateful
features of CakeML to avoid repeated evaluation of the same thunk values.

The operational semantics of lazy CakeML will be written as a part of the
project. As the semantics of CakeML is specified in higher-order logic
(HOL)~\cite{HOLInter57:online},
the operational semantics for CakeML will be translated from HOL to Haskell
functions. The compiler will then connect the lazy and strict semantics.

Tests will be written to show that the compiler generates semantically
equivalent code. This project will employ manual unit testing to test
expressions and steps of evaluation.

In summary, the project consists of three major parts that were developed
in parallel:
\begin{itemize}
 \item Writing lazy semantics for the programming language CakeML
 \item Creating a compiler that translates the lazy semantics to the original strict semantics of CakeML
 \item Testing the compiler by writing operational semantics in the form of functions and running unit tests.
\end{itemize}

\section{Limitations}
\label{sec:limitations}
Due to the time constraints of this thesis, certain limitations will be enforced
to keep the workload within a reasonable size. These limitations are:

\paragraph{Only expressions were considered.}
As expressions make up of the majority of the language, they will be the highest
category of the grammar that will be implemented. Expressions allow operations
and values to be represented, which is important for a programming language.
This limitation means that declarations and modules will not be a part of this
project.

\paragraph{No parser was developed for the compiler.}
No actual parser that takes written CakeML code was created for this thesis.
Instead, an internal representation of CakeML expressions was created in
Haskell and used to create the semantics and compiler. A parser that
takes CakeML code in string format and converts it to the internal
representation can be created in the future. It should not prove to
be a challenging task to link the parser together with the compiler.

\paragraph{A fixed version of CakeML was focused on.}
At the time of writing, the programming language CakeML is still evolving,
meaning that the semantics of the language is prone to change at
any time. In order to keep a steady workflow, any updates to the semantics after
the start of this thesis will not be considered.
