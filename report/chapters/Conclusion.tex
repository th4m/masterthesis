\chapter{Conclusion}
This thesis presented a Haskell definition of CakeML's strict semantics,
a Haskell definition of lazy semantics for CakeML's pure parts, and a Haskell
implementation of a compiler that compiles lazy CakeML to strict CakeML.

The first step of this project, translating CakeML's semantics from Lem to
Haskell, was performed successfully. Three items were transcribed to Haskell:

\begin{itemize}
\item The abstract syntax of CakeML.
\item The helper functions for CakeML's semantics.
\item The top level interpreter function \texttt{evaluate}.
\end{itemize}
The abstract syntax represents the grammar of the language, including literals,
expressions, and operations. The highest level of the grammar that was
translated was expressions, as it is the most important part of the language
that allows operations and literals to be represented.
The semantic primitives of a language are the operations that are needed
to go produce a value. Some of these operations include arithmetics and
pattern matching. Also a part of the semantic primitives are the representation
of values, environment, and state.
In order to go from an expression to a value, the \texttt{evaluate} function
links the abstract syntax and the semantic primitives. Taking a state,
a environment, and a list of expressions, \texttt{evaluate} applies the
operations found in the semantic primitives to produce a state and a value.

The similarity in syntactical structure between the Lem and Haskell made the
task of translating CakeML's semantics allowed the translation to be finished
begin fairly early, which was followed by the second phase of the project:
defining the lazy semantics. The task of defining the lazy semantics mainly
consisted of creating a new definition of \texttt{evaluate}, called
\texttt{evaluateLazy}, that targets the pure parts of CakeML.
The lazy semantics used the idea of thunks to suspend
evaluation of expressions until called. Thunks were defined under the value
datatype in the semantic primitives and contained an expression together with
its environment. The function \texttt{evaluateLazy} tries to evaluate an
expression as little as possible before returning either a thunk with the
partially evaluated expression and its environment, or another type of value.
When a thunk is \textit{forced} to become a usable value, with a function
appropriately called \texttt{force}, its expression is repeatedly evaluated with
its environment until it produces a usable (non-thunk) value.

When the definition of the lazy semantics was finished, the third phase of this
project began: constructing the compiler. This was formalised with a
\texttt{compile} function that takes an expression and alters it with respect to
the lazy semantics defined for this project. As such, thunks were defined by
using the \texttt{Con} expression, which represents constructors. Together with
the \texttt{Thunk} constructor, a \texttt{Val} constructor was also created to
represent all non-thunk values. Values were simply put inside the \texttt{Val}
constructor, while thunks required more processing. To simulate the
encapsulation that thunks had in the lazy semantics, an expression was put
inside a \texttt{Fun} expression to be evaluated to closures. This allows the
expression to stay together with its environment in the \texttt{Thunk}
constructor. In order for the compiled expressions to evaluate to usable values,
a \texttt{force} function was defined. The idea behind the compiler's
\texttt{force} was the same as the one defined for the lazy semantics, except
that the compiler's \texttt{force} was defined by using CakeML expressions.
By doing this, \texttt{force} could be injected into the compiled code to be
evaluated together with the thunks and values, allowing non-thunk values to be
produced at the end of the evaluation chain.

The implementation of the compiler allowed evaluation of expressions to be
delayed until when needed. However, delaying the evaluation only results in
\textit{call-by-name} semantics, meaning that if the thunks are used on multiple
occasions, they will be evaluated each time. True laziness is achieved through
\textit{call-by-need} semantics where thunks retain the value that was yielded
by the evaluation of their expressions. This optimisation was implemented by
altering how thunks were produced and how \texttt{force} handled this new
definition of thunks. Thunks were changed to contain a reference to a location
in the state, where a suspended expression would be stored. Thus, \texttt{force}
was change to extract the value that the reference pointed to, which would be
represented by one of two new constructors \texttt{RefExp} and \texttt{RefVal}.
Suspended expressions were represented by \texttt{RefExp}, while the values that
the expressions were forced to were represented by \texttt{RefVal}. The change
from \texttt{RefExp} to \texttt{RefVal} was handled by using an assign operation
whenever a \texttt{RefExp} was forced to replace the value that the thunk
reference was pointing at.

The ultimate goal of this project was to produce a compiler that takes lazy
semantics and enforces it in strict semantics. This goal has
been reached with the completion of the third phase. The call-by-need semantics
optimised the compiler to achieve true laziness, where forced thunks
would replace their suspended expressions with the value yielded from
the evaluation.
