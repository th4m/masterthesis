\chapter{Strict Semantics}
\label{chapter:strict}
As stated in section~\ref{intro:project}, the first major step of this project
was to write lazy semantics for CakeML in Haskell. However, before this could be
done, the existing strict semantics would have to be defined. The definition
would include the basic abstract syntax tree and semantic primitives of the
language.

\section{Implementation}
As CakeML is an already defined language, its underlying semantics already
exists. The compiler for this project is to be written in Haskell, meaning that
the semantics must be defined in Haskell in order for it to be usable. This is
what this section will describe: defining the strict semantics of CakeML in
Haskell.

The semantics of CakeML is defined in Lem, as described in
section~\ref{back:cakeml}. At the beginning of this thesis, a link to CakeML's
GitHub repository was provided, in which the semantics was contained in the form
of \textit{.lem} files. The goal of this part of the thesis was thus to manually
translate the semantics from Lem to Haskell. 

In order to translate the semantics of CakeML, three items needed to be
investigated and manually transcribed from Lem to Haskell:
\begin{itemize}
\item The abstract syntax of CakeML
\item The semantic primitives of CakeML
\item The \texttt{evaluate} function
\end{itemize}

\noindent All three items were defined as Lem modules in the provided GitHub
repository and needed to be defined as Haskell modules.
Details about the contents and implementation of these three items
will be described in the following sections.

\subsection{Abstract Syntax}
\label{strict:abs}
The abstract syntax is the basic syntactical representation of the grammar of a
language~\cite{pltbook}. This includes the representation of literals,
operations, and expressions. Translating the abstract syntax from Lem to Haskell
proved to not be all too difficult, as Lem and Haskell are fairly similar in
structure. With the use of algebraic datatypes, each category in the grammar is
defined as its own type. For example, some of the expressions of CakeML
are represented in Lem as such:

\begin{figure}[H]
\begin{alltt}
  type exp =
    | Lit of lit
    | Var of id varN
    | Fun of varN * exp
    | App of op * list exp
\end{alltt}
\end{figure}

\noindent In order from top to bottom, these expressions represent literals,
variables, functions, and applications. The constructors are complemented with
data from other datatypes, e.g. \texttt{lit} and \texttt{id}, that are defined
as a part of the language.
The above expressions were translated to Haskell as such:

\begin{figure}[H]
\begin{alltt}
  data Exp
    = Literal Lit
    | Var (Id VarN)
    | Fun VarN Exp
    | App Op [Exp]
\end{alltt}
\end{figure}

\noindent Similarly, complementing datatypes, such as \texttt{Lit} and
\texttt{id}, as well as other categories in the grammar, were also translated
as needed.

These expressions are the basic building blocks that a programmer uses to create
their program. For example, a simple integer such as 5 is represented as
\begin{alltt}
  Literal (IntLit 5)
\end{alltt}
when using this abstract syntax. Building more
complex expressions is as simple as combining multiple expressions. For example
the arithmetic expression $1 + (7 - 8)$ can be represented as

\begin{figure}[H]
\begin{alltt}
  App (OPN Plus) [Literal (IntLit 1),
                  App (OPN Minus) [Literal (IntLit 7)
                                  ,Literal (IntLit 8)
                                  ]
                 ]
\end{alltt}
\end{figure}

\subsection{Semantic Primitives}
The semantic primitives of a language is how its calculations are performed.
This includes operations such as basic arithmetic, logical operations,
lookups, and pattern matching. Also included in the semantic primitives are
the inner representations of \textit{values}, \textit{state}, and
\textit{environment}.

Values are represented by the datatype \texttt{V} and are what expressions
ususally are evaluated to. Some of the values of CakeML are represented in
Lem as such:

\begin{alltt}
  type v =
    | Litv of lit
    | Closure of environment v * varN * exp
    | Vectorv of list v
\end{alltt}

\noindent In order, these values represent literals, closures, and vectors.
Just as with the expressions described above, other datatypes that represent
other parts of the grammar are needed. However, these mostly already exist from
the definition of the abstract syntax and are therefore reused.

The translation of these values to Haskell was performed analogously to the
abstract syntax described in~\ref{strict:abs}:

\begin{figure}[H]
\begin{alltt}
  data V
    = LitV Lit
    | Closure (Environment V) VarN Exp
    | VectorV [V]
\end{alltt}
\end{figure}

An example of an expression and its value counterpart is the expression $1 + 2$,
whose result is the value 3. The expression is represented as
\begin{alltt}
  App (OPN Plus) [Literal (IntLit 1), Literal (IntLit 2)]
\end{alltt}
which is evaluated to the value
\begin{alltt}
  LitV (IntLit 3)
\end{alltt}

% Should state and environment be described? Too much info?
% Just refer to appendix for more info?

\subsection{The \texttt{evaluate} function}
Linking the abstract syntax together with the semantics of CakeML is the
\texttt{evaluate} function. The job of \texttt{evaluate} is to take a list of
expressions as input and return a state together with a result. A result can
consist of a list of values or an error. Also contributing to the evaluation
of expressions are a state and an environment, which are given as input to
\texttt{evaluate}. The type of \texttt{evaluate} (in Lem notation) is thus:

\begin{alltt}
evaluate:state -> environment v -> list exp -> state * result (list v) v
\end{alltt}

\noindent Expressions are taken either as a sequence in \texttt{evaluate},
in which case they are evaluated sequentially, or as singletons, in which case
they are pattern matched to be evaluated with unique strategies:

\begin{figure}[H]
\begin{alltt}
  let rec
  evaluate st env []           = ...
  and
  evaluate st env (e1::e2::es) = ...
  and
  evaluate st env [Lit l]      = ...
  and
  ...
\end{alltt}
\end{figure}

\noindent The logic of each case mostly consists of applying appropriate
operations from the semantic primitives. For example, when evaluating
the \texttt{Var n} expression, the \texttt{lookup\_var\_id} function defined
in the semantic primitives is applied to \texttt{n} in order to check if
a value is assigned to the variable in the environment. In the case it does
exist, the value is returned:

\begin{figure}[H]
\begin{alltt}
  evaluate st env [Var n] =
    match lookup_var_id n env with
    | Just v -> (st, Rval [v])
    | Nothing -> (st, Rerr (Rabort Rtype_error))
    end
\end{alltt}
\end{figure}

\noindent Expressions that contain subexpressions (e.g. binary arithmetic
operations) often require that one or more of these subexpressions are
evaluated to values in order to use them as input for the primitive semantics
functions. For example, the expression \texttt{If e1 e2 e3} requires that
\texttt{e1}, which is the condition, is evaluated first in order to use the
semantic function \texttt{do\_if} to decide whether \texttt{e2} or
\texttt{e3} should be evaluated.

%This function has been modified to omit fix_clock, which isn't used for this project
\begin{figure}[H]
\begin{alltt}
  evaluate st env [If e1 e2 e3] =
    match st (evaluate st env [e1]) with
    | (st', Rval v) ->
        match do_if (head v) e2 e3 with
        | Just e -> evaluate st' env [e]
        | Nothing -> (st', Rerr (Rabort Rtype_error))
        end
    | res -> res
    end
\end{alltt}
\end{figure}

The translation to Haskell maintained the same structure of \texttt{evaluate}
with similar usage of lists and pattern matching.

\begin{figure}[H]
\begin{alltt}
  evaluate st env (e1:e2:es)  = ...
  evaluate st env [Var n]     = ...
  evaluate st env [Literal l] = ...
\end{alltt}
\end{figure}

\noindent The logic of \texttt{evaluate} was also maintained in the translation
to Haskell. For example, the \texttt{Var n} expression is defined in Haskell as
follows:

\begin{figure}[H]
\begin{alltt}
  evaluate st env [Var n] =
    case lookup_var_id n env of
      Just v  -> (st, RVal [v])
      Nothing -> (st, RErr (RAbort RType_Error))
\end{alltt}
\end{figure}

\section{Testing the Implementation}
%% Tested the implementation
%% Created a Haskell module to run tests
%% Imported all three above mentioned modules
%% Created dummy state and environment
%% Created a function that uses the state and environment, and takes expressions as input
%%   The function runs evaluate with the dummy state/environment and returns state + result
%% Created a number of test expressions to run through evaluate
%% Compared the result of evaluate with expected values


\section{Discussion/Conclusion}
