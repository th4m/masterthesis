\chapter{Strict semantics of CakeML}
\label{chapter:strict}
As stated in Section~\ref{intro:project}, the first major step of this project
is to write lazy semantics for CakeML in Haskell. However, before this can be
done, the existing strict semantics must be defined in Haskell. The definition
includes the basic abstract syntax tree and semantic primitives of the
language.

\section{Definition}
CakeML is a language with a semantics defined prior to this project.
The compiler for this project is written in Haskell and produces CakeML
syntax in the form of its Haskell representation. This means that
the semantics must be defined in Haskell in order for it to be compatible with
the code that the compiler produces. This is
what this section will describe: defining the strict semantics of CakeML in
Haskell.

The semantics of CakeML is defined in Lem~\cite{Lem33:online}, as described in
Section~\ref{back:cakeml}. At the beginning of this project, a link to CakeML's
GitHub repository was provided, containing the semantics in the form
of \textit{.lem} files. The goal of this part of the thesis is thus to manually
translate the semantics from Lem to Haskell.

In order to translate the semantics of CakeML, three items must be
investigated and manually transcribed from Lem to Haskell:
\begin{itemize}
\item The abstract syntax of CakeML
\item Helper functions for CakeML's semantics
\item The top level interpreter function \texttt{evaluate}
\end{itemize}

\noindent All three items are defined as Lem modules in the provided GitHub
repository and need to be defined as Haskell modules.
Details about the contents and definition of these three items
will be described in the following sections.

\subsection{Abstract syntax}
\label{strict:abs}
The abstract syntax is the basic syntactical representation of the grammar of a
language~\cite{pltbook}. This includes the representation of literals,
operations, and expressions. As Lem and Haskell are fairly similar in
structure, translating the abstract syntax from Lem to Haskell is not an all too
difficult task. With the use of algebraic datatypes, each construct in the grammar is
defined as its own type. For example, some of the expressions of CakeML
are represented in Lem in Figure~\ref{fig:lemexp}.
The constructors are complemented with data from other datatypes,
e.g. \texttt{lit} and \texttt{op}, that are also defined as a part of the
language. These datatypes are shown in Figures~\ref{fig:lemlit}
to~\ref{fig:lempat}.

\begin{figure}
\begin{alltt}
  type exp =
    (* Literal *)
    | Lit of lit
    (* Variable *)
    | Var of id varN
    (* Anonymous function *)
    | Fun of varN * exp
    (* Application of operators to arguments *)
    | App of op * list exp
    (* Pattern matching *)
    | Mat of exp * list (pat * exp)
    (* Let expression that (potentially) binds a value to a variable
       in the environment and evaluates another expression *)
    | Let of maybe varN * exp * exp
    (* Local definition of (potentially) mutually recursive functions *)
    | Letrec of list (varN * varN * exp) * exp
    ...
\end{alltt}
\caption{Some expressions of CakeML defined in Lem.}
\label{fig:lemexp}
\end{figure}

\begin{figure}
\begin{alltt}
  (* Literal constants *)
  type lit =
    | IntLit of integer
    | Char of char
    | StrLit of string
    ...
\end{alltt}
\caption{Some literals of CakeML defined in Lem.}
\label{fig:lemlit}
\end{figure}

\begin{figure}
\begin{alltt}
  (* Operators *)
  type op =
    (* Operations on integers *)
    | Opn of opn
    | Opb of opb
    (* Function application *)
    | Opapp
    (* Reference operations *)
    | Opassign
    | Opref
    | Opderef
    ...


  (* Arithmetic operators *)
  type opn = Plus | Minus | Times | Divide | Modulo

  (* Comparison operators *)
  type opb = Lt | Gt | Leq | Geq
\end{alltt}
\caption{Some operators of CakeML defined in Lem.}
\label{fig:lemop}
\end{figure}

\begin{figure}
\begin{alltt}
  (* Identifiers *)
  type id 'a =
    | Short of 'a
    | Long of modN * 'a


  (* Variable names *)
  type varN = string

  (* Module names *)
  type modN = string

  (* Constructor names *)
  type conN = string

\end{alltt}
\caption{Identifiers and some names of CakeML defined in Lem.}
\label{fig:lemid}
\end{figure}

\begin{figure}
\begin{alltt}
  (* Patterns *)
  type pat =
    | Pvar of varN
    | Plit of lit
    | Pcon of maybe (id conN) * list pat
    ...
\end{alltt}
\caption{Some patterns of CakeML defined in Lem.}
\label{fig:lempat}
\end{figure}

The Lem definition of CakeML expressions in Figure~\ref{fig:lemexp} is
translated to Haskell in Figure~\ref{fig:hsexp}.
Similarly, complementing datatypes, such as \texttt{lit} and
\texttt{op}, as well as other constructs in the grammar, are also translated
as needed. These datatypes can be seen in Figure~\ref{fig:hsdata}.

\begin{figure}
\begin{alltt}
  data Exp
    -- Literal
    = Literal Lit
    -- Variable
    | Var (Id VarN)
    -- Anonymous function
    | Fun VarN Exp
    -- Application of operators on arguments
    | App Op [Exp]
    -- Pattern matching
    | Mat Exp [(Pat, Exp)]
    -- Let expression that (potentially) binds a value to a variable
    -- in the environment and evaluates another expression
    | Let (Maybe VarN) Exp Exp
    -- Local definition of (potentially) mutually recursive functions
    | LetRec [(VarN, VarN, Exp)] Exp
    ...
\end{alltt}
\caption{Haskell definition of some CakeML's expressions.}
\label{fig:hsexp}
\end{figure}

\begin{figure}
\begin{alltt}
  -- Literal constants
  data Lit
    = IntLit Int
    | Char Char
    | StrLit String
    ...


  -- Operators
  data Op
    -- Integer operations
    = OPN Opn
    | OPB Opb
    -- Function application
    | OpApp
    -- Reference operations
    | OpAssign
    | OpRef
    | OpDeref
    ...

  -- Arithmetics operators
  data Opn = Plus | Minus | Times | Divide | Modulo

  -- Comparison operators
  data Opb = Lt | Gt | LEq | GEq


  -- Identifiers
  data Id a = Short a | Long ModN a


  -- | Patterns
  data Pat
    = PVar VarN
    | PLit Lit
    | PCon (Maybe (Id ConN)) [Pat]
    ...
\end{alltt}
\caption{Haskell definitions of some complementing CakeML datatypes.}
\label{fig:hsdata}
\end{figure}

\newpage
\noindent These expressions are the basic building blocks that a programmer uses
to create their programs. For example, a simple integer such as 5 is represented
as
\begin{alltt}
  Literal (IntLit 5)
\end{alltt}
when using this abstract syntax. Building more
complex expressions is as simple as combining multiple expressions. For example,
with the \texttt{Fun} expression, it is possible to create functions that take
arguments and produces values. For example, \texttt{take} that is described in
Section~\ref{intro:Example} can be defined by combining a number of expressions.
The definition of \texttt{take} using CakeML expressions in Haskell can be seen
in Figure~\ref{fig:caketake}.

\begin{figure}
\begin{alltt}
  cakeTake =
    LetRec [("take", "n",
             Fun "ls" $
             If (App Equality [Var (Short "n"), Literal (IntLit 0)])
              (Con (Just (Short "nil")) [])
              (Mat (Var (Short "ls"))
               [(PCon (Just (Short "::" )) [PVar "elem", PVar "rest"]
                ,Con (Just (Short "::")) [Var (Short "elem")
                                         ,cakeTake2])
               ,(PCon (Just (Short "nil")) []
                ,Con (Just (Short "nil")) [])]
              )
            )] $
    Var (Short "take")
    where cakeTake2 =
            App OpApp [App OpApp [Var (Short "take")
                                 ,decr]
                      ,Var (Short "rest")]
          decr =
            App (OPN Minus) [Var (Short "n")
                            ,Literal (IntLit 1)]
\end{alltt}
\caption{The recursive function \texttt{take} defined by combining CakeML expressions in Haskell.}
\label{fig:caketake}
\end{figure}


\subsection{Helper functions for CakeML's semantics}
CakeML's semantics has many helper functions that define how computations are
performed. This includes operations such as basic arithmetic, logical
operations, lookups, and pattern matching. Also included in the semantic
primitives are the inner representations of \textit{values}, \textit{state}, and
\textit{environment}.

Values are represented by the datatype \texttt{v} (in Haskell it is denoted with
an upper case \texttt{V}) and are what expressions
usually are evaluated to. Some of the values of CakeML are represented in
Lem as seen in Figure~\ref{fig:vallem}.

\begin{figure}
\begin{alltt}
  type v =
    (* Literal values *)
    | Litv of lit
    (* Function closures *)
    | Closure of environment v * varN * exp
    (* Function closure for recursive functions *)
    | Recclosure of environment v * list (varN * varN * exp) * varN
    ...
\end{alltt}
\caption{Some values of CakeML defined in Lem.}
\label{fig:vallem}
\end{figure}

\noindent In order, these values represent literals, closures, and recursive closures.
Just as with the expressions described above, other datatypes that represent
other parts of the grammar are needed. However, these mostly already exist from
the definition of the abstract syntax and are therefore reused.

The translation of these values to Haskell is performed analogously to the
abstract syntax described in Section~\ref{strict:abs} as seen in
Figure~\ref{fig:valhas}.

\begin{figure}
\begin{alltt}
  data V
    -- Literal values
    = LitV Lit
    -- Function Closures
    | Closure (Environment V) VarN Exp
    -- Function closure for recursive functions
    | RecClosure (Environment V) [(VarN, VarN, Exp)] VarN
    ...
\end{alltt}
\caption{Some values of CakeML defined in Haskell.}
\label{fig:valhas}
\end{figure}

An example of an expression and its value counterpart is the expression $1 + 2$,
whose result is the value 3. The expression is represented as
\begin{alltt}
  App (OPN Plus) [Literal (IntLit 1), Literal (IntLit 2)]
\end{alltt}
which is evaluated to the value
\begin{alltt}
  LitV (IntLit 3)
\end{alltt}
The helper function used to actually apply the addition to the literals
is \texttt{do\_app},
which pattern matches on the given operation and arguments and
uses another helper function \texttt{opn\_lookup} to find the \texttt{(+)}
operator:
\begin{alltt}
  do_app s op vs =
    case (op, vs) of
      (OPN op, [LitV (IntLit n1), LitV (IntLit n2)]) ->
        Just (s, RVal (LitV (IntLit (opn_lookup op n1 n2))))
  ...

  opn_lookup n =
    case n of
      Plus   -> (+)
  ...
\end{alltt}


\subsection{The top level interpreter function \texttt{evaluate}}
Linking the abstract syntax to the semantics of CakeML is the
\texttt{evaluate} function. The job of \texttt{evaluate} is to take a list of
expressions as input and return a state together with a result. A result can
consist of a list of values or an error. Also contributing to the evaluation
of expressions are a state and an environment, which are given as input to
\texttt{evaluate}. The type of \texttt{evaluate} (in Lem notation) is thus:

\begin{alltt}
  evaluate:state -> environment v -> list exp -> state*result (list v) v
\end{alltt}

\noindent Expressions are taken either as a list in \texttt{evaluate},
in which case they are evaluated sequentially, or as singletons, in which case
they are pattern matched to be evaluated with unique strategies:
\newpage
\begin{alltt}
  let rec
  evaluate st env []           = (st, Rval [])
  and
  evaluate st env (e1::e2::es) = ...
  and
  evaluate st env [Lit l]      = (st, RVal [Litv l])
  and
  ...
\end{alltt}

\noindent The logic of each case mostly consists of applying appropriate
operations from the semantic helper functions. For example, when evaluating
the \texttt{Var n} expression, the \texttt{lookup\_var\_id} helper function
is applied to \texttt{n} in order to check if
a value is assigned to the variable in the environment. In the case it does
exist, the value is returned:

\begin{alltt}
  evaluate st env [Var n] =
    match lookup_var_id n env with
    | Just v -> (st, Rval [v])
    | Nothing -> (st, Rerr (Rabort Rtype_error))
    end
\end{alltt}

\noindent Expressions that contain sub-expressions (e.g. binary arithmetic
operations) often require that one or more of these sub-expressions are
evaluated to values in order to use them as input for the semantic helper
functions. For example, the evaluation of the expression \texttt{If e1 e2 e3}
requires that
\texttt{e1}, which is the condition, is evaluated first in order to use the
semantic function \texttt{do\_if} to decide whether \texttt{e2} or
\texttt{e3} should be evaluated.

\begin{alltt}
  evaluate st env [If e1 e2 e3] =
    match st (evaluate st env [e1]) with
    | (st', Rval v) ->
        match do_if (head v) e2 e3 with
        | Just e -> evaluate st' env [e]
        | Nothing -> (st', Rerr (Rabort Rtype_error))
        end
    | res -> res
    end
\end{alltt}

The translation to Haskell maintains the same structure of \texttt{evaluate}.
The type declaration for \texttt{evaluate} is transcribed to Haskell as:
\begin{alltt}
  evaluate :: State -> Environment V -> [Exp] -> (State, Result [V] V)
\end{alltt}

\noindent Lists and pattern matching are used in similar ways in the Haskell definition:

\begin{alltt}
  evaluate st env []          = (st, RVal [])
  evaluate st env (e1:e2:es)  = ...
  evaluate st env [Literal l] = (st, RVal [LitV l])
\end{alltt}

\noindent The logic of \texttt{evaluate} is also maintained in the translation
to Haskell. For example, the \texttt{Var n} expression is defined in Haskell as
follows:
\newpage
\begin{alltt}
  evaluate st env [Var n] =
    case lookup_var_id n env of
      Just v  -> (st, RVal [v])
      Nothing -> (st, RErr (RAbort RType_Error))
\end{alltt}

\section{Example of evaluation}
\label{strict:example}
By using \texttt{evaluate}, CakeML expressions are evaluated to values. This
section will show how such an expression is evaluated by using the definition of
\texttt{take} described in Section~\ref{intro:Example} and shown in abstract
syntax at the end of Section~\ref{strict:abs}.

When \texttt{take} is evaluated, the first grammatical construct that is
encountered is \texttt{LetRec} that creates a \texttt{RecClosure} when evaluated.
When the \texttt{RecClosure} is applied to an integer literal with the use of the
\texttt{App OpApp} expression, the literal will be bound to the variable \texttt{n}.
The expression representing \texttt{take} in the \texttt{RecClosure} is a
\texttt{Fun}, which is evaluated to a \texttt{Closure} that takes a list
\texttt{ls}. When the \texttt{Closure} is applied to a list with \texttt{App OpApp},
the actual logic that is described in
Section~\ref{intro:Example} will be carried out. When the \texttt{If} expression
is evaluated, the condition is evaluated first: \texttt{n} is compared with 0 to
check if they are equal. If they are equal, the constructor \texttt{nil} is
returned by the \texttt{do\_if} helper function, which will
be evaluated to a constructor value.
If \texttt{n} is not equal to 0, the second expression is returned by
\texttt{do\_if}, which pattern matches on the list \texttt{ls} with the
\texttt{Mat} expression to check if it is
either a \texttt{::} or \texttt{nil} constructor. As \texttt{ls} is evaluated
for this check, it will be fully evaluated by the evaluation steps of the
\texttt{Con} expression.
In the case of \texttt{::},
the head element is stored as the variable \texttt{elem}, and the tail of the
list is stored as the variable \texttt{rest}. Then a \texttt{Con} expression
for a list is created with \texttt{elem} as its head element and the application
of \texttt{take} on \texttt{n} subtracted by 1 and the tail of the list
\texttt{rest}. In the case of \texttt{nil} being the constructor of \texttt{ls},
a \texttt{nil} constructor is simply returned, ending the computation.


\section{Testing the definition}
To make sure that CakeML's semantics are properly translated to Haskell, testing
is performed.
Tests consist of running \texttt{evaluate} on a number of predefined
expressions together with dummy instances of state and environment.
The results of the evaluations are then compared with expected values
in the form of unit testing to check for any deviations.

Faulty results indicate erroneous implementation of the semantics.
Through repeated troubleshooting and amending code, the semantics translation
has reached a satisfactory state that evaluate expressions correctly,
allowing the next phase of the project to begin: defining the lazy semantics.
