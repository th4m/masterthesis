\chapter{Future work}

This chapter will discuss potential work that could be performed in
addition to what has been done for this project. Suggestions will not only
concern what the compiler takes as input and produces, but also tasks such as
testing.

\section{Compiling declarations}
As stated in Section~\ref{sec:limitations}, expressions are the topmost
construct of CakeML's grammar considered for this project. CakeML has a level
of grammar above expressions: declarations. Declarations handle
features such as multiple
bindings at once and type definitions, and increase the expressiveness of the
language. In order to support all of CakeML's semantics, defining declarations
in the Haskell port of the semantics must be done. Lazy semantics and
compilation of declarations would then need to be defined and implemented.

\section{Property-based testing}
For this project, the semantics and compiler are tested by using unit tests.
As unit tests are dependent on the manual work of inventing and writing
test cases, they cannot be fully trusted to be able to catch all types of
errors. In the case of compiler construction, the expressiveness of the
source and target languages can be tremendous, making it very difficult to
think of all types of expressions to test. A different approach to testing
is property-based testing. For Haskell, QuickCheck is a well known tool for
formulating and running property-based
tests~\cite{Claessen:2000:QLT:351240.351266}.
It was suggested at the planning phase of this project that QuickCheck would
be used to run property-based tests for the produced compiler. However, doing
property-based testing on compilers was deduced to be a task that may warrant
a project of its own. Thus, due to time constraints, property-based testing
was not included for this project.

\section{Update semantics}
CakeML is a language that is being developed at the time of
writing. As such, the version of semantics used to create the compiler for this
project was already outdated shortly after the project start. If the compiler
is to be used for the newest version of CakeML, the semantics would need
updating. For constant support, the semantics would need to be continuously
updated.

\section{Applying the compiler to actual CakeML code}
This project only handled the intermediate representation of CakeML's abstract
syntax. An interesting application of the compiler would be to create a parser
to produce code that can be executed through CakeML's own compiler. The parser
would need to be able to take CakeML's concrete syntax, convert it to the
Haskell port of the abstract syntax. Once the abstrax syntax has been generated,
it can be run through the compiler to produce altered abstract syntax, which
can then be converted back to the concrete syntax.


% Application to actual CakeML code
% Update the compiler to the latest version of CakeML
