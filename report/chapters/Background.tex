\chapter{Background}
Chapter~\ref{chapter:intro} shortly introduced the basics of lazy and strict
semantics. This chapter will present a more in-depth description of semantics
and compilers, as well as present some previous work on the subject.

\section{Semantics}
Every programming language has its own \textit{semantics}. Semantics are often
informally explained in English prose, but some languages have formal semantics.
The semantics of a programming language defines
the meaning of programs~\cite{Fernand:PLangOpSem}.
In other words, the semantics of a language describes how programs are executed.
Different languages tend to use the same constructs, e.g. \texttt{if/else}, but
evaluate these constructs differently. This is where the semantics defines the
language.

The semantics of a language can be defined in different ways. The most
natural approach for the semantics of this project is
\textit{operational semantics}. Fernández states that operational semantics
is very useful for implementing languages and proving correctness of compiler
optimisations~\cite{Fernand:PLangOpSem}. By using operational
semantics, each construct is defined by computation steps. These steps are
often described with the help of transition systems, presenting the computation
steps as a sequence of transitions. For this project, however,
the operational semantics will be defined in the form of Haskell functions.


\subsection{Strict Semantics}
Strict programming indicates that expressions are always fully evaluated when
they appear in the chain of evaluation. For example, when an arithmetic expression
is bound to a variable, the expression is fully evaluated to become a numerical
value before being assigned to the variable. The value is then accessible
by simply using the variable.

\subsection{Lazy Semantics}
\label{back:lazySem}
The main characteristics of laziness is delaying evaluation of expressions
until they are needed. This is often done when binding expressions to
variables. In order to delay the evalution of an expression, lazy semantics
makes use of a data structure called
\textit{thunk}~\cite{Ennals:2003:OEA:944746.944731}. A thunk contains a
suspended expression together with information that is required to finish the
evaluation of the expression. Expressions that are to be
evaluated are suspended in so called thunk values and passed on correspondingly.
When needed, these thunks are \textit{forced} in order to yield an actual usable
value.
% Using this
%mechanic requires more memory usage than when evaluating with strict semantics.

\subsection{Pros and Cons of Both Semantics}
One reason to why applications are programmed strictly is because it can be
necessary to have all values fully evaluated. If the evaluation of an expression
that is erroneous is delayed, it's impendent error will also be delayed.
An example would be to consider a stateful system; if an operation is dependent
on a reference, the value assigned to that reference needs to have been
evaluated. If the evaluation of that value is delayed until after it is used,
an error is prone to occur, as that value will not be there.
By using strict evaluation, the evaluation is not delayed, avoiding any
ambiguity of knowing whether or not references will have valid values.

It is not always certain that all expressions that are contained in a program
will be used every time that it is executed. The more complex a program is, the
higher the chance is for branching. For example, a number of expressions may
be evaluated and bound to variables before a chain of
\texttt{if}/\texttt{else if} expressions. Certain branches of this program may
only use a select few, or even none, of the previously assigned variables.
In such cases, it would not be optimal to actually evaluate them.

Lazy evalution can be motivated with an opposite reason in relation to strict
evaluation. Programs that have a number of expressions bound by variables may
not make use of all of them. For example, when using the \texttt{case} construct
to pattern match in Haskell, the expression that is pattern matched can be
assigned to a variable as an effect of the pattern matching:

\begin{figure}[H]
\begin{alltt}
  case list of
    [] -> ...
    xs -> ...
\end{alltt}
\end{figure}

\noindent In the second case, \texttt{list} is assigned to the variable
\texttt{xs} as an effect of the pattern matching. For the pattern matching,
\texttt{list} is evaluated to check its outermost constructor. If it is an empty
list, it will matched to the \texttt{[]} case, while the \texttt{xs} case
handles all other cases, where \texttt{list} is bound to the variable
\texttt{xs}.

The evaluation of
\texttt{list} is delayed until \texttt{xs} is called and used, meaning
that the actual content of list can be infinite in length without having the
evaluation of the program ending up in a non-terminating loop.

Infinite data structures are also a major benefit for using lazy evaluation.
As described in section~\ref{intro:Example}, the function \texttt{take}
in Haskell returns a subset of a given list. \texttt{take} has two arguments,
an integer \texttt{x} and a list \texttt{xs}. This is used to get the
$x$ first elements of the list \texttt{xs}. The lazy evaluation of the
\texttt{xs} in \texttt{take} allows the list to be infinite in size. This
can be used for a program that e.g. returns the first \texttt{x} elements of
the Fibonacci numbers.


\section{Compilers}
When a programmer has written code, the code must be translated into a form such
that it can be executed by a computer. This translation is the job of a
\textit{compiler}~\cite{DragonBook}. The compiler is a program that reads
programs in a \textit{source} language and translates it to a \textit{target}
language. The source language is often some high-level language, while the
target language is a low-level machine language.
For example, when compiling programs with Java as source language, the target
language that the compiler translates the code to is most likely Java Virtual
Machine (JVM)~\cite{JavaJVM}.

There exists a certain category of compilers that translates code from a
high-level language to another high-level language. This category of compilers
is often referred to as \textit{transpilers} or
\textit{source-to-source compilers}~\cite{kulkarnitranspiler}.
The goal of transpilers is to generate code in the target language that behaves
equivalently as the code in the source language. The compiler defined in this %Should this be here?
thesis can be considered to be of this category. In this case, the source
language is lazy CakeML and the target language is strict CakeML.

A common practice of compilers is to create an intermediate representation of
the input code in the form of constructors in the form of a syntax
tree~\cite{DragonBook}. The intermediate representation needs to be
easy to produce and is used for optimising the code, as well as allowing for
easy generation of code in the target language.
For this project, the intermediate representation of CakeML expressions is %Should this be here?
written as abstract syntax trees (ASTs) in Haskell. The ASTs will be altered so
that when the target CakeML code is generated, it should be able to be evaluated
lazily, but with the use of the native CakeML compiler that evaluates strictly.

\section{CakeML}
\label{back:cakeml}
This section will present CakeML, the language that is the main focus of this
thesis. \notdone{TODO}

\section{Previous Work}
This section will present and discuss previous work that is related to this
project. Topics include programming language libraries that allow lazy/strict
evaluation in languages that do not allow it by default, as well as an
evaluation strategy for Haskell that tries to optimise the inherent laziness
of the language.

\subsection{Libraries}
There are many languages that are strict that have additional libraries
containing lazy versions of primitive functions and vice versa. This section
will present some examples of languages with this type of functionality.

\subsubsection{Haskell}
Haskell is one of very few languages that are inherently lazy. All primitive
operations are lazy by nature, but the language also has certain functions that
enforce strict evaluation.

An example of a function that enforces strictness is \texttt{seq}. \texttt{seq}
takes two arguments: \texttt{a} and \texttt{b}, and returns \texttt{b}. The
result of \texttt{seq} is dependent on both \texttt{a} and \texttt{b}. If
\texttt{a} does not terminate, \texttt{seq} does not terminate. If \texttt{a}
does terminate, the result of \texttt{seq} is equal to \texttt{b}. While both
\texttt{a} and \texttt{b} are evaluated before returning the result,
\texttt{seq} does not guarantee that \texttt{a} is evaluated before \texttt{b}.

\begin{itemize}
\item \$!\notdone{TODO}
\end{itemize}

\subsubsection{.NET}\notdone{TODO}

\subsubsection{Python}\notdone{TODO}

\subsection{Optimistic evaluation}

This section will present and discuss an evaluation strategy that was developed
for Haskell by Ennals and Peyton Jones. In their paper they state that they
measure common usage of thunks. This lead them to create an evaluation strategy
that makes use of their observation. \notdone{Expand more.}

