\thispagestyle{empty}

\begin{abstract}
  The evaluation strategies of programming languages can be broadly categorised as strict or lazy.
  A common approach to strict evaluation is to implement a call-by-value semantics
  that always evaluates expressions when they are bound to variables, while lazy
  evaluation is often implemented as call-by-need semantics that evaluates
  expressions when they are needed for some computation. Lazy semantics makes
  use of thunks, a data structure that contains an expression whose evaluation
  has become suspended together with its environment. This thesis presents
  (1) a Haskell definition of the existing semantics of CakeML, a strict programming
  language, (2) a Haskell definition of lazy semantics for the pure part
  of CakeML, and (3) a Haskell implementation of a compiler that compiles lazy
  CakeML to strict CakeML with respect to (1) and (2). The compiler will make
  use of stateful features in strict CakeML to optimise evaluation so that each
  thunk is evaluated at most once, simulating a call-by-need semantics.
\end{abstract}
