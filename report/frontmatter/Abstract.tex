\thispagestyle{empty}

\begin{abstract}
  Evaluation strategies can vary significantly between programming languages.
  Two commonly known strategies are strict and lazy evaluation.
  A common approach to strict evaluation is to implement a call-by-value semantics
  that always evaluates expressions when they are bound to variables, while lazy
  evaluation is often implemented as call-by-need semantics that evaluates
  expressions when they are needed for some computation. Lazy semantics makes
  use of thunks, a data structure that contains an expression whose evaluation
  has become suspended together with its environment. This thesis presents
  (1) a translation of the strict semantics of CakeML, a strict programming
  language, to Haskell, (2) a definition of lazy semantics for CakeML, and
  (3) a compiler that that makes use of the lazy semantics to reconstruct
  expressions in order to make the strict evaluation in CakeML become lazy.
\end{abstract}
